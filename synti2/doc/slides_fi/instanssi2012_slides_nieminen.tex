% -*- coding: utf-8 -*-

\documentclass{beamer}
\usepackage{amsfonts}
\usepackage{amsmath}

\mode<presentation>
{
  \usetheme{Hannover}
}

\usepackage[finnish]{babel}
\usepackage[utf8]{inputenc}

\usepackage{times}

%\graphicspath{{figs/}}

\title% [](optional, use only with long paper titles)
{Alle 1k (tai sinne päin) softasyntikka}

\author{Paavo Nieminen}

\date{Instanssi 2012, Voionmaan lukio, 9.3.2012}

\begin{document}

\begin{frame}
  \titlepage
\end{frame}

\begin{frame}{Outline}
  \tableofcontents
  % You might wish to add the option [pausesections]
\end{frame}

\section{Syntetisaattori}

\begin{frame}{Softasyntikka}
\begin{itemize}
  \item Ääni: korvalla aistittu ilman värähtely, värähtelylähteestä
    kuten soittimesta
  \item Värähtelyn voi tuottaa kaiutin sähkösignaalin ohjaamana
  \item sähkösignaali tallennettu äänilähteestä kuten soittimesta
  \item sähkösignaali voi olla myös syntetisoitu eli synnytetty ilman
    mekaanista värähtelijää; silloin signaalin tuottajaa sanotaan
    syntetisaattoriksi
  \item syntetisaattori voi olla tietokoneohjelma.
\end{itemize}
\end{frame}

\section{Miksi? Miksi, miksi miksi?}
\begin{frame}{Miksi? Miksi, miksi miksi?}
  \begin{itemize}
  \item Miksi uusi softasyntikka? Onhan noita jo!
  \item Mutta mikään niistä ei ole ''omatekemä''!
  \item Eikä ''pikkiriikkisiä'' avoimen lähdekoodin syntsia ole liian
    paljon juossut vastaan.
  \item \ldots ``Oma on aina oma'', totesi nörtti, ja alkoi koodata.
  \end{itemize}
\end{frame}


\begin{frame}{Pikkiriikkiset demot}
\begin{itemize}
  \item Tähän projektiin valitsemani harrastusgenre on ``4k introt''.
  \item Miksi?
  \item Miksi ei? Miksi esimerkiksi harrastetaan aitajuoksua, vaikka
    juokseminen on helpompaa ilman esteitä? Miksi kolmiloikka, kun jo
    neljällä loikalla pääsisi pidemmälle?
  \item Parempi ilmeisesti olla kysymättä tyhmiä toisen harrastuksesta
    :). Minä teen nyt tätä ``old-school 4k matskua''.
\end{itemize}
\end{frame}

\begin{frame}{Pikkiriikkisyys}
\begin{itemize}
\item ''Pikkiriikkinen'' tarkoittaa esim. että käynnistettävän (ja
  toivottavasti toimivan) ohjelman koko on ''pieni'' eli esimerkiksi
  alle 4096 tavua.
\item Osin ''huijausta'' - pakatut exet ym. ''temput''.
\item Ei enempää tempuista tässä. Näistä on kerrottu mm. viime
  Instanssissa; samat temput ovat tietysti käytössä tässäkin
  projektissa.
\item Tässä puuhassa pääsee kivasti miettimään, miten ohjelma mahtuu
  pienempään tilaan -- mukavaa kuten ristisanat tai sudokut.
\item Rajoitteet nykypäivänä keinotekoisia -- miksi esim. tehdä
  C-kielellä, kun Javalla/Javascriptillä saa pienempään tilaan
  hienomman?. Miksi esim. äänisynteesin pitää olla osa koodia eikä
  ulkopuolisesti linkitetty kirjasto?
\item Valitsin tällä kertaa huvikseen rajoitteiksi mm. C-kielen +
  linkityksen ainoastaan SDL:ää ja OpenGL:ää vasten.
\end{itemize}
\end{frame}

\begin{frame}{Softasyntikan toimintaympäristö}
  \begin{itemize}
  \item DA-muunnin, digitaalisen datan puskuri äänikortissa,
    ajuriohjelmistot, käyttöjärjestelmä
  \item Kirjasto-ohjelmat järjestelmän abstrahoimiseksi
  \item Ohjelmoijan rajapinta: aika-ikkuna täytettäväksi numeroilla,
    esim. 256 tavua, joista ulostulosignaalin amplitudi määräytyy
    esim. 48000 kertaa sekunnissa kanavaa kohti. Ikkuna on täytettävä
    salamannopeasti, jotta ''ääni ei pätki''.
  \item ''Musiikkia'' varten pitää jostain tulla myös tieto rytmistä,
    äänenkorkeudesta ja -väristä sekä harmoniasta. Eli
    ''instrumentit'' ja ''nuottien sekvenssi'' ilmeisesti tarvitaan.
  \item Eli tavoite on hyvin selkeä. Sitte eiku koodaamaan.
  \end{itemize}
\end{frame}

\section{Ekat purkat: Synti}
\begin{frame}{tästä se lähti}
\begin{itemize}
\item Teinistä asti haikaillut koodailla näitä, mutta ei tullut
  aloitettua\ldots
\item Innostukseni heräsi uudelleen Zipolan 4k introsta ekoilla
  Instansseilla pari vuotta sitten. ''Mäkin haluun tehdä meteliä!''
\item Mitäs meillä pittäis olla? Varmaan fraasisekvensseri, jotain
  oskillaattoreita, verhokäyriä, filttereitä, \ldots ja kiva nimi
  projektille.
\item Niinpä ''tein Syntiä'' pari kuukautta ennen Instanssia 2011.
\item Syntyi ekat purkkaviritykset \ldots
\item Hyvät tuli.
\item Meteliä! Jes!!!
\end{itemize}
\end{frame}


\begin{frame}{Mutta..}
\begin{itemize}
\item
  Musiikkia hankala editoida (tekstimuotoinen ``oma formaatti''\ldots
  tyhmä idea). Ajattelin säveltää ekan biisin jälkeen myös toisen,
  mutta kunto loppui syötedatan muotoiluun ekan bassoriffin kohdalla.
\item
  Melko pikkiriikkistä koodia, mutta ei kovin nopeata -- ei
  reaaliaikaisesti soiteltavissa.
\item
  Kuvan synkronointia äänen kanssa ei ollut tullut ajateltua yhtään
  alunperin -- vain purkka ja päälleliimaus olivat mahdollisia
  varsinaisen 4k intron tekemisen hetkellä.
\item
  Sounditkin oli vähän niin ja näin, vaikka vahingossa toteutuneella
  ''köyhän miehen taajuusmodulaatiolla'' sai ihan riittävän pahvisen
  kumivirvelin.
\item
  Koodissa jopa niin hämmästyttävän paljon purkkaa, ettei kehdannut
  julkaista.
\end{itemize}
\end{frame}


\begin{frame}{Kantapääopetuksia}
\begin{itemize}
\item
  Numero 1: Älä keksi MIDIä uudestaan; se on jo ihan hyvä
  (lähtökohdaksi, ei lopputulemaksi).
\item
  Siis kertaus: älä keksi uudestaan protokollaa, jonka
  insinööriarmeija on jo miettinyt läpi.
\item
  Numero 2: Valmiit työkalut (esim. sekvensseriohjelmat) ovat ihan
  hyviä ja niitä on mietitty. Hyödynnä!
\item
  Numero 3: Syntikka on kivampi reaaliaikaisena.
\item
  Numero 4: Koodinnälkä kasvaa koodatessa, jos innostuu jostain
  puuhasta.
\end{itemize}
\end{frame}


\section{Uudet purkat: Synti2}
\begin{frame}{Synti2:n vaatimukset 1/2}
\begin{itemize}
\item Kalusto tuottaa pikkiriikkisen suoritettavan ohjelman, jossa on
  mukana ääni ja grafiikka (so. ennakkoon kiinnitetty musiikillinen ja
  graafinen sekvenssi sekä syntesoidut
  instrumenttiäänet). Suoritettava ohjelma datoineen on kooltaan alle
  4096 tavua, ja sen voi suorittaa standardissa käyttöympäristössä
  (kehitysvaiheessa 32-bittinen Linux, libSDL, libGL, gunzip TAI
  64-bittinen Linux johon asennettuna 32-bittiset SDL \& OpenGL)
\item
  Grafiikan parametrit voidaan lukea syntetisaattorin tilasta, jota
  sekvenssi ohjaa. (Äänitapahtumiin synkronoitu animaatio)
\item
  Sävellysvaiheessa synteesi on reaaliaikainen. Latenssin on oltava
  alle 10 millisekuntia (jota suuremmat latenssit häiritsevät jopa
  meikäläistä rähmätassua sunnuntaisoittelijaa)
\end{itemize}
\end{frame}

\begin{frame}{Synti2:n vaatimukset 2/2}
\begin{itemize}
\item Sävellysvaiheessa ohjattavissa standardilla
  MIDIllä. Mahdollistaa musiikin (ja tarkemmin ajatellen myös
  animaation ohjaussekvenssin) tuottamisen millä tahansa
  MIDI-sekvensseriohjelmistolla.
\item Suoritettavan ohjelman sekvenssi muodostetaan (karsien ja
  muuntaen) standardimuotoisesta MIDI-musiikkitiedostosta (SMF0 tai
  SMF1).
\item Instrumentteja varten on parametrieditori, jossa jokainen
  parametri on muutettavissa hiirellä raahaamalla. Äänien tallennus ja
  lataus tapahtuu MIDI-maailmassa tavanomaisella SysEx-tiedostojen
  käytänteellä.
\item Äänenlaadultaan vähintään yhtä hyvä kuin alkuperäinen ``Synti
  ykkönen''
\item Lisävaatimus Instanssin aikataulun kirjoittajalta:
  synteesimoduulin koko on alle 1 kilotavu. Tämä oli ainoa, joka ei
  aivan täysin toteutunut :).
\end{itemize}
\end{frame}

\begin{frame}{Toteutuksesta}
\begin{itemize}
\item
  FM-synteesi, 4 siniaalto-operaattoria
\item
  Wavetablet, Verhokäyrät, Filtterit ja Delayt kuten edellisessä,
  joskin aavistuksen verran paremmin
\item 
  Sekvensseri
\item 
  MIDI-rajapinta
\item
  Apuohjelmat: reaaliaikainen MIDI-adapteri, offline
  SMF-adapteri, soundieditori.
\item
  Huhhuh!! Aika paljon sitten putkahtikin tekemistä pikkiriikkisen
  oskillaattorin ympärille \ldots
\end{itemize}
\end{frame}


\begin{frame}{Tilanne 9.3.2012 klo 20:38}
  \begin{itemize}
  \item
    Aika kiva lelu \ldots
  \item
    Katsotaanpa kalustoa livenä \ldots Demoefektin todennäköisyys oli
    kuin olikin suuri (\ldots yllättäen oli kuitenkin muu kuin oma
    softa, joka jumitti audiosysteemin.)
    \begin{itemize}
    \item Jack käyntiin
    \item Syntikka käyntiin
    \item ''MIDI-mukiloija'' käyntiin
    \item Äänieditori käyntiin
    \item Koskettimet kiinni ja jamittelemaan \ldots
    \end{itemize}
  \item Biisin säveltäminen vapaavalintaisella
    MIDI-sekvensserillä. Esim. Rosegarden on oikeasti hyvä ohjelma,
    vaikka oli huteralla tuulella tänään.
  \item Paketointi ''4k introksi'' on helppoa kuin 'make tinyexe'.
  \item Eli miten kävi otsikon lupaukselle ''alle 1k softasyntikka''?
    Synamoduulin saa runtattua 1.6 kiloon, mutta äänen saaminen
    äänikortista ulos asti (SDL:n kautta) pullahtaa minimissään 2
    kiloon.
  \end{itemize}
\end{frame}

\begin{frame}{Kantapääopetuksia}
\begin{itemize}
\item Vois lukea matematiikkaa silloin, kun siihen on oikeasti aikaa,
  vaikkapa perusopintojen yhteydessä. Signaaliprosessoinnin teoria
  (suodatinsuunnittelu ym.) helpottaisi näitä harrastuksia kovasti.
\item
  Vois koodata kaikenlaista useammin, ettei taidot ruostu esim. olio-,
  grafiikka- tai käyttöliittymäohjelmoinnin osalta.
\item
  Vois vaikka tämän jälkeen päästää irti menneisyydestä ja siirtyä
  muihin kuin C-kieleen.
\end{itemize}
\end{frame}

\begin{frame}{Lähdekoodit ja Workshoppailu Instansseilla}
  \begin{itemize}
  \item Lähdekoodit MIT-lisenssillä noin puoleen yöhön mennessä
    (Malttia!): \url{https://yousource.it.jyu.fi/synti2}
  \item Apuohjelmat vielä bugisia ja muutenkin hiukan
    vaiheessa. Kivoja ovat ikuinen silmukka reaaliaikasäikeessä tai
    täysillä korvaan tuleva valkoinen kohina. Varokaa, jos kokeilette.
  \item Parantunevat (koodit, ei korvat), jahka Instanssin kompojen
    deadlinet lähestyvät :)
  \item Nykikää hihoista, jos aihepiiri(t) kiinnostaa enempi.
  \end{itemize}
\end{frame}

\end{document}


