% -*- coding: utf-8 -*-

\documentclass{beamer}
\usepackage{amsfonts}
\usepackage{amsmath}

\mode<presentation>
{
  \usetheme{Hannover}
}

\usepackage[finnish]{babel}
\usepackage[utf8]{inputenc}

\usepackage{times}
% Hmm.. utf8?
%\usepackage[T1]{fontenc}


\graphicspath{{figs/}}

\title% [](optional, use only with long paper titles)
{Alle 1k softasyntikka}

%\author[Author, Another] % (optional, use only with lots of authors)
%{F.~Author\inst{1} \and S.~Another\inst{2}}
% - Give the names in the same order as the appear in the paper.
% - Use the \inst{?} command only if the authors have different
%   affiliation.
\author{Paavo Nieminen}
%\institute{}

%\institute[Universities of Somewhere and Elsewhere] % (optional, but mostly needed)
%{
%  \inst{1}%
%  Department of Computer Science\\
%  University of Somewhere
%  \and
%  \inst{2}%
%  Department of Theoretical Philosophy\\
%  University of Elsewhere}
% - Use the \inst command only if there are several affiliations.
% - Keep it simple, no one is interested in your street address.

\date{Instanssi 2012}
% - Either use conference name or its abbreviation.
% - Not really informative to the audience, more for people (including
%   yourself) who are reading the slides online

%\subject{Theoretical Computer Science}
% This is only inserted into the PDF information catalog. Can be left
% out. 



% If you have a file called "university-logo-filename.xxx", where xxx
% is a graphic format that can be processed by latex or pdflatex,
% resp., then you can add a logo as follows:

%\pgfdeclareimage[height=1.8cm]{university-logo}{mit_logo}
%\logo{\pgfuseimage{university-logo}}



% Delete this, if you do not want the table of contents to pop up at
% the beginning of each subsection:
%\AtBeginSubsection[]
%{
%  \begin{frame}<beamer>{Outline}
%    \tableofcontents[currentsection,currentsubsection]
%  \end{frame}
%}

%\AtBeginSection[]
%{
%  \begin{frame}<beamer>{Outline}
%    \tableofcontents[currentsection]
%  \end{frame}
%}



% If you wish to uncover everything in a step-wise fashion, uncomment
% the following command: 

%\beamerdefaultoverlayspecification{<+->}


\begin{document}

\begin{frame}
  \titlepage
\end{frame}

\begin{frame}{Outline}
  \tableofcontents
  % You might wish to add the option [pausesections]
\end{frame}


% Structuring a talk is a difficult task and the following structure
% may not be suitable. Here are some rules that apply for this
% solution: 

% - Exactly two or three sections (other than the summary).
% - At *most* three subsections per section.
% - Talk about 30s to 2min per frame. So there should be between about
%   15 and 30 frames, all told.

% - A conference audience is likely to know very little of what you
%   are going to talk about. So *simplify*!
% - In a 20min talk, getting the main ideas across is hard
%   enough. Leave out details, even if it means being less precise than
%   you think necessary.
% - If you omit details that are vital to the proof/implementation,
%   just say so once. Everybody will be happy with that.


\section{Johtanto}

\subsection{Mitäs sitten oikein tulikaan tehtyä?}

\subsection{Miksi tein sen \ldots}
\begin{frame}{Miksi? Miksi, miksi miksi?}

\end{frame}


\begin{frame}{Softasyntikka}
\begin{itemize}
  \item Ääni-ilmiö kuten musiikki: Tajunta, korva, ilman värähtely,
    värähtelyn lähde kuten soitin
  \item Värähtelyn voi tuottaa kaiutin sähkösignaalin ohjaamana
  \item sähkösignaali tallennettu äänilähteestä kuten soittimesta
  \item signaali voi olla myös syntetisoitu
  \item Syntetisaattori voi olla tietokoneohjelma.
  \item Ympäristö: DA-muunnin, digitaalinen puskuri, ajuriohjelmistot
  \item Rajapinta: aika-ikkuna täytettäväksi numeroilla, esim. 256
    tavua.
\end{itemize}
\end{frame}

\subsection{Ekat purkat: Synti Software Synthesizer}
\begin{frame}{tästä se lähti}
Ja nälkä kasvoi syödessä.
\end{frame}

\begin{frame}{Kantapääopetuksia}

  Numero 1: Älä keksi MIDIä uudestaan; se on jo ihan hyvä.

  Siis älä keksi uudestaan valmiiksi mietittyä protokollaa.

  Numero 2: Jos voit käyttää valmiita työkaluja, käytä.

  Numero 3: Syntikka on kivampi reaaliaikaisena.

\end{frame}


\subsection{Synti2}
\begin{frame}{Vaatimukset}
  Tuotettavissa pikkiriikkinen suoritettava ohjelma, jossa on mukana
  ääni (so. musiikillinen sekvenssi syntesoiduin instrumentein) ja
  grafiikka. Kokonaisuus datoineen oltava alle 4096 tavua.

  Grafiikan parametrit voidaan asettaa syntetisaattorin tilan
  perusteella, sekvenssin ohjaamana. (Ääneen synkronoidut animaatiot)

  Sävellysvaiheessa reaaliaikainen. Latenssin on oltava alle 10
  millisekuntia (jota suuremmat latenssit häiritsevät jopa meikäläistä
  rähmätassua sunnuntaisoittelijaa)

  Sävellysvaiheessa ohjattavissa standardilla MIDIllä. Mahdollistaa
  musiikin (ja tarkemmin ajatellen myös animaation ohjaussekvenssin)
  tuottamisen millä tahansa MIDI-sekvensseriohjelmistolla.

  Suoritettavan ohjelman sekvenssi muodostetaan (karsien ja
  suodattaen) standardimuotoisesta MIDI-musiikkitiedostosta (SMF0 tai
  SMF1).

  Instrumentteja varten on parametrieditori, jossa jokainen parametri
  on muutettavissa hiirellä raahaamalla. Äänien tallennus ja lataus
  tapahtuu MIDI-maailmassa tavanomaisella SysEx-tiedostojen
  käytänteellä.

  Kuulostaa vähintään yhtä hyvältä kuin alkuperäinen ``Synti ykkönen''

  [Lisävaatimus Instanssin aikataulun kirjoittajalta: synteesimoduulin
    koko on alle 1 kilotavu.]

\end{frame}

\begin{frame}{Toteutuksesta}
  FM-synteesi

  Wavetablet

  Filtterit

  Delayt

  Sekvensseri

  Apuohjelmat. Huhhuh!!
  
\end{frame}


\begin{frame}{Lopputulema}
  Aika kiva lelu tuli!
\end{frame}

\begin{frame}{Kantapääopetuksia}

  Vois lukea matematiikkaa silloin, kun siihen on oikeasti resursseja.

  Vois koodata kaikenlaista useammin, ettei taidot ruostu esim. olio-,
  grafiikka- tai käyttöliittymäohjelmoinnin osalta.

  Vois päästää irti menneisyydestä ja siirtyä
  funktio-ohjelmointiin. ``Ai että olis taas tämäkin [Asia X] helppoa,
  kun ei tarttis aloittaa for(p=st; *p != 0; p++)\{...''
  
\end{frame}

\begin{frame}{Lähdekoodit ja lisenssi}
  YouSource-linkki tähän.
\end{frame}

\begin{frame}{Workshoppia Instanssissa}
  Kysymyksiä ja keskustelua tästä tai muista softasyntikoista tai
  ylipäätään äänisynteesistä...
\end{frame}


\end{document}


