% -*- coding: utf-8 -*-

\documentclass[pdf,10pt]{beamer}
\usepackage{amsfonts}
\usepackage{amsmath}

\mode<presentation>
{
  \usetheme{Goettingen}
  \usecolortheme{seagull}
}

\setbeamertemplate{navigation symbols}{}
\setbeamertemplate{bibliography item}{[\theenumiv]}

\setlength{\parskip}{\baselineskip}


\usepackage[finnish]{babel}
\usepackage[utf8]{inputenc}

\usepackage{palatino}

%\graphicspath{{figs/}}

\title[Linux 1k intro -- ``making of'']
{Yhden kilotavun intro natiivina (x86-64) Linux-binäärinä. Hullulla on huvinsa?}

\author{Paavo Nieminen}

\date{Instanssi 2016, Valorinne, Jyväskylä, 4.-6.3.2016}

\begin{document}

\begin{frame}
  \titlepage
\end{frame}

%\begin{frame}{Sisältö}
%{
%\setlength{\parskip}{\smallskipamount}
%  \tableofcontents
%  % You might wish to add the option [pausesections]
%}
%\end{frame}
\beamerdefaultoverlayspecification{<+->}

\section{Sisältö}
\begin{frame}{1k Intro Linux-binäärinä -- ``Making of'':}
  \begin{itemize}
  \item Taustaa: Demoscenen koodarikisat ``pikkiriikkisille
    ohjelmille'' (kokorajoitus esim. 4096, 1024 tai 256 tavua).
    %Miksi ihmeessä?
  \item Mikä nyt taas lähti lapasesta (+miksi), ja mitä siitä opittiin?
  \item Kaikki tulee käytännön kautta. Screenille lävähtää mm.:
    \begin{itemize}
    \item AMD64-konekieltä, ELF-binääritiedostoa, POSIX-shell
      -komentoja ja deflate-pakattua dataa ennen ja jälkeen
      pakkauksen. Heksalukuina, luonnollisesti\ldots
    \item Assembler-lähdekoodia, GLSL-varjostinkieltä ja SDL2:n ja
      OpenGL:n funktionimistöä. ASCII-tekstinä, luonnollisesti.
    \item Deflate-pakattua bittivirtaa juuri sellaisena kuin se tulee
      ulos tämän hetken ehkä parhaasta pakkausohjelmasta (zopfli).
    \item Alkuperäisen datavirran jokaisen tavun pakattu bittimäärä.
      Bittien sommittelu voi koukuttaa enemmän kuin Sudokut!
    \item Rikkeitä, rikoksia ja rötöksiä standardeja vastaan kohdissa,
      joissa ``valvonta pettää'' nykyisessä Linuxissa.
    \end{itemize}
  \item Lopussa kerätään yhteen, miksi tämä näpertely kuitenkin ehkä
    jälleen kannatti\ldots
  \item Erityisesti tietotekniikan opiskelijoille pari täkyä tulevien
    kurssien, harkkojen ja tutkimusaiheiden suhteen!
  \end{itemize}
\end{frame}



\section{Demoscene ja Introt}
\subsection{Demoscene}
\begin{frame}{Mikä se Demoscene ja ``Intro'' on?}
  \begin{itemize}
  \item Historia alkaa noin 1980-luvulta, kun ``tietotekniikkaa
    paremmin osaava nuoriso'' purki pelien kopiosuojauksia
    (laittomasti) ja jakeli näitä ns. ``kräkättyjä'' pelejä.
  \item Laittomassa jakeluversiossa oli tyypillistä laittaa alkuun
    näkyville omat terveiset kräkkerikavereille ja huomautella omasta
    paremmuudesta, tyyliin ``oltiin ekat!''.
%    mm. siinä kuinka nopeasti
%    laillisen pelin julkaisun jälkeen oli saatu oma ilmainen, laiton,
%    versio liikkeelle.
  \item Terveisten muoto oli tietokoneohjelma, joka käynnistyy ennen
    itse peliä tuottaen grafiikkaa ja ääntä, joilla viesti
    välittyy. Siinä mielessä kyseessä oli ``Intro'' eli esittely
    kräkätylle pelille ja ryhmän eli ``groupin'' saavutukselle.
  \item Introstakin haluttiin mahdollisimman hieno ja näyttävä.
  \item Tietokoneet olivat rajallisia, kotikäyttöön myytäviä
    kasarivehkeitä, samoin kuin jakeluvälineet (C-kasetit,
    disketit). Jotta itse peli mahtuisi mukaan, intron koodin
    ja datan täytyi mahtua todella pieneen tilaan.
  \item Itse en vielä ollut mukana, mitä etenkin laittomuuden osalta
    haluan korostaa. Hymiö tähän. Olin tuossa vaiheessa vielä aika
    junnu\ldots päädyin mukaan vasta laillisemmassa vaiheessa (about
    1994).
\end{itemize}
\end{frame}

\begin{frame}{Ihminen haluaa kilpailla?}
  \begin{itemize}
  \item Paitsi itse kräkkäämisen osalta, tietenkin piti
    ``elvistellä'' kavereille siinä, miten hieno intro oli tullut
    tehtyä\ldots
%  \item Ja kun muutenkin oli ryhdytty pitämään miittejä\ldots
  \item Lähtivätten sitten erikseen kilpailemaan siinä, kuka osasi
    tehdä hienoimman näköisiä juttuja tietokoneella.
  \item Kun kerran kilpailtiin, ``perustettiin laji'' ja luotiin
    säännöt.
  \item Hommasta tuli myös taidetta sinällään, luomisen muoto.
  \item ``Rajoittamattomassa'' sarjassa (``demo'') näytettiin, mitä
    kaikkea osattiin tehdä.. joskus hienommin kuin saman aikakauden
    tietokonepeleissä. Saattoivatpa demoryhmät myös päätyä tekemään
    pelejä, joissa demoissa demoillut tekniikat pääsivät käyttöön.
  \item ``Rajoitetut'' sarjat mallinsivat alkuperäisiä introja, joissa
    suoritettavan tiedoston koko oli teknisten syiden vuoksi
    pieni. Toki sarjoja oli erikseen myös mm. grafiikoille ja
    musiikoille. Instanssi jatkaa genren perinnettä hienosti!
  \item Siinä lyhyesti. Mm. Yle Areenasta löytyy näköjään YLE:n ja
    Alternative Partyn tuottama dokumenttisarja
    ``Demoscene-dokumentti'' (Demoscene Documentary), josta
    kannattaapi tsekata, mitä oikein tapahtui:
    \url{http://areena.yle.fi/1-1299532}
  \end{itemize}
\end{frame}

\subsection{Nykypäivä}
\begin{frame}{Mistä on kyse nykyään?}
  \begin{itemize}
  \item Digitaalista taidetta.
  \item Kilpailuissa mahdollisuus mitata omaa osaamista ja näyttää
    sitä muille. Tapahtumissa myös vaihdetaan ja jaetaan vinkkejä.
  \item Tuotos on joka tapauksessa aina taideteos, josta tulee osa
    alakulttuurin historiaa, riippumatta kilpailusijoituksesta.
  \item Eli demoparty on aina vähän niinkuin Euroviisut?
  \item Demoissa tarvittavat taidot ovat suoraan käytettävissä
    esimerkiksi peliteollisuudessa ja mediatuotannossa.
  \item ``Rajoittamaton sarja'' eli ``demo'' tarjoaa nykyään upeita
    ja/tai mielenkiintoisia useimmiten ryhmätyönä toteutettuja
    tuotoksia, joissa musiikki, grafiikka, 3D-mallinnus, tarina,
    tuotanto, projektityöskentely ja koodaus yhdistyvät.
  \item Rajoitetuissa sarjoissa on edelleen useimmiten rajoitteena
    suoritettavan ohjelmatiedoston koko. Esim. Assembly -tapahtumassa
    kilpaillaan nykyään 4096 tavun ja 1024 tavun
    sarjoissa. Instanssissa ``pikkiriikkisen'' raja on 4096.
  \end{itemize}
\end{frame}

\subsection{Miksi ihmeessä?}
\begin{frame}{Miksi pitää tehdä asiasta niin vaikeaa?}
  \begin{itemize}
  \item Miksi pitää tehdä esim. 4096 tai 1024 tavun tiedostoja, kun
    1000000 tavua olisi niin paljon helpompi?
  \item Hei\ldots Se on harrastus! Vertaa aitajuoksu: Minkä takia
    pitää kasata jotain aitoja siihen matkalle, kun olisi helpompaa
    juosta ilman aitoja?
  \item Kuten juoksussa vs. aitajuoksussa, demokisoissakin korostuu
    erilaiset taidot ja vahvuudet eri sarjoissa.
  \item Rajoitetuimpia sarjoja voitaneen sanoa ``koodarin
    yksilökilpailuiksi'', koska mukaan ei yksinkertaisesti mahdu
    paljonkaan ``joukkueen'' tuottamaa grafiikkaa, musiikkia tai
    muutakaan, mitkä taas ovat välttämätön edellytys kovempitasoisen
    täysmittaisen demon tekemisessä.  Paljolti kaikki on generoitava
    ohjelmallisesti ajon aikana.
  \item Rajoitekoon alittaminenkin on haaste, saati sitten mielekkään
    ja/tai mielenkiintoisen kokonaisuuden luonti.
  \item Onneksi vasta-alkajalla on tänä päivänä helppoa, koska
    valmiita runkoja on saatavilla hyvin, ja oman koodaamisen voi
    keskittää taiteelliseen puoleen eikä kivuliaaseen runkokoodin koon
    pienentämiseen\ldots
  \item[\ldots] josta kivusta tämä esitys kylläkin kertoo :)
  \end{itemize}
\end{frame}

\section{1k intro Linuxille}
\begin{frame}{Miksi 1k intro on jotenkin ``vaikea''?}
  \begin{itemize}
  \item Tässä on 1024 merkkiä (x-kirjaimia esimerkin vuoksi):{\footnotesize \\
    xxxxxxxxxxxxxxxxxxxxxxxxxxxxxxxxxxxxxxxxxxxxxxxxxxxxxxxxxxxxxxxx\\%
    xxxxxxxxxxxxxxxxxxxxxxxxxxxxxxxxxxxxxxxxxxxxxxxxxxxxxxxxxxxxxxxx\\%
    xxxxxxxxxxxxxxxxxxxxxxxxxxxxxxxxxxxxxxxxxxxxxxxxxxxxxxxxxxxxxxxx\\%
    xxxxxxxxxxxxxxxxxxxxxxxxxxxxxxxxxxxxxxxxxxxxxxxxxxxxxxxxxxxxxxxx\\%
    xxxxxxxxxxxxxxxxxxxxxxxxxxxxxxxxxxxxxxxxxxxxxxxxxxxxxxxxxxxxxxxx\\%
    xxxxxxxxxxxxxxxxxxxxxxxxxxxxxxxxxxxxxxxxxxxxxxxxxxxxxxxxxxxxxxxx\\%
    xxxxxxxxxxxxxxxxxxxxxxxxxxxxxxxxxxxxxxxxxxxxxxxxxxxxxxxxxxxxxxxx\\%
    xxxxxxxxxxxxxxxxxxxxxxxxxxxxxxxxxxxxxxxxxxxxxxxxxxxxxxxxxxxxxxxx\\%
    xxxxxxxxxxxxxxxxxxxxxxxxxxxxxxxxxxxxxxxxxxxxxxxxxxxxxxxxxxxxxxxx\\%
    xxxxxxxxxxxxxxxxxxxxxxxxxxxxxxxxxxxxxxxxxxxxxxxxxxxxxxxxxxxxxxxx\\%
    xxxxxxxxxxxxxxxxxxxxxxxxxxxxxxxxxxxxxxxxxxxxxxxxxxxxxxxxxxxxxxxx\\%
    xxxxxxxxxxxxxxxxxxxxxxxxxxxxxxxxxxxxxxxxxxxxxxxxxxxxxxxxxxxxxxxx\\%
    xxxxxxxxxxxxxxxxxxxxxxxxxxxxxxxxxxxxxxxxxxxxxxxxxxxxxxxxxxxxxxxx\\%
    xxxxxxxxxxxxxxxxxxxxxxxxxxxxxxxxxxxxxxxxxxxxxxxxxxxxxxxxxxxxxxxx\\%
    xxxxxxxxxxxxxxxxxxxxxxxxxxxxxxxxxxxxxxxxxxxxxxxxxxxxxxxxxxxxxxxx\\%
    xxxxxxxxxxxxxxxxxxxxxxxxxxxxxxxxxxxxxxxxxxxxxxxxxxxxxxxxxxxxxxxx\\%
}
    \item Tuon verran tilaa olis sulla ohjelmakoodille, joka tuottaa
      jotain mielekästä audiovisuaalista tulostetta.
    \item Ja koodin pitää siis olla ``klikkaamalla'' käynnistettävissä
      ja suoritettavissa jollain yleisellä alustalla, esim. Windows,
      Mac, Linux, Web-selain.
  \end{itemize}
\end{frame}

\subsection{Erilaiset alustat}
\begin{frame}{Erilaiset alustat 1/2}
  \begin{itemize}
    \item Perusmuodossaan edes ``Hello world'' -ohjelma ei nykyään
      taivu alle 1024 tavun muussa kuin tulkattavassa ympäristössä,
      jossa koko ohjelma kuuluu print "Hello world!"\ tmv. (21
      merkkiä, mukaanlukien lopussa oleva
      rivinvaihto). Tulkattavillakin grafiikan ja audion tuotto on
      haastavaa saada mahtumaan pieneen tilaan.
    \item Pääsääntöisesti 1k introt ovat pakattuja, jolloin ohjelman
      käynnistämisen jälkeen mahdollisimman lyhyellä (välttämättä
      pakkaamattomalla) koodilla ensin avataan varsinainen intro, joka
      sitten suoritetaan.
    \item Selainpuolella Javascript myös tyypillisesti lyhennetään
      automaattisilla työkaluilla, jolloin lähdekoodin sanat
      ``piirraKolmio(kolmioLista[i],nykyinenVari);'' voivat muuttua
      esim. muotoon ``a(b[d],c);''. Onhan se lyhyempi, mutta ei tuossa
      muodossa ollenkaan kehitettävissä. Siksi tarvitaan
      työkaluohjelmat, ns. ``minifier'' -ohjelmat.
    \item Vastaava automaattinen pienennys tehdään grafiikassa
      tarvittaville varjostimille eli shadereille. Ja lopuksi kaikki
      vielä pakataan (Web-alustoilla yleensä PNG-kuvaksi, olennaisesti
      siis deflate-algoritmilla).
  \end{itemize}
\end{frame}

\begin{frame}{Erilaiset alustat 2/2}
  \begin{itemize}
    \item Nykyään suosittu alusta on Web-selain, jossa on
      käyttöjärjestelmän päällä monipuolinen ja hyvin standardoitu
      välikerros. Tuotokset ovat luonnostaan käyttöjärjestelmästä
      riippumattomia. Grafiikkaa voi tehdä WebGL:llä, eli OpenGL:n
      Web-versiolla, tai suoraan selaimen kanvaasielementillä.
    \item Toinen suosittu alusta on Windows, jolle on saatavilla aika
      hyvät introntekijän työkalut Visual C:n päälle. Osa näitäkin on
      automaattinen pakkaaminen ja avaaminen.
    \item Macillä on myös tehty näyttäviä 1k-tuotoksia. Itselläni ei
      toistaiseksi ole tarkempaa tietoa Mac-maailman työkaluista,
      mutta evidenssin perusteella niitä täytyy olla saatavilla hyvin.
    \item Linux-alustalle itselläni ei ole tiedossa varsinaisia
      valmiita työkaluja. Linuxille on ollut tosi vähän 1k-tuotoksia
      viime vuosina -- esim. Assemblyssä käsittääkseni 4 kpl koskaan
      toistaiseksi (2015 asti)
    \item Joten tuota\ldots\ Mikä lähti lapasesta tällä kertaa?
    \item Piti ihan ite väsätä Linux 1k ja kahtoa mitä siihen
      liittyy\ldots
  \end{itemize}
\end{frame}

\begin{frame}{Reunahuomautus}
  \begin{itemize}
    \item Oletko kiinnostunut aloittamaan harrastuksen 4k tai 1k
      introjen parissa?
    \item Miedosti suosittelen aloittamaan Web-alustan päällä eli
      Javascript \& WebGL \& valmiit ohjelmistot
      pienentämiseen. Ohessa oppii väistämättä nykypäivänä keskimäärin
      hyvin relevanttia Web-ohjelmointia.
    \item[]
      {\tiny  \url{http://www.p01.org/} (Assembly 2015 voittajan sivusto aiheesta),\\
      %mm. 47 minuutin video, jossa Assembly
      %2015 1k:n voittaja duunaa livenä audiovisuaalisen demon
      %selaimelle;
      \url{http://creativejs.com/2012/06/jsexe-javascript-compressor/}
      (JS-pienentäjäsofta Windowsille)\\
      \url{https://gist.github.com/gasman/2560551} (Simppeli malli
      Ruby-kielellä)\\
      }
      
    \item Jos natiivisti haluat tehdä, niin Windowsilla tai Macillä
      päässee helpommalla. Löytyy enemmän ja hienompia aiempia
      tuotoksia, joiden päälle rakentaa omaa taidetta.
    \item Osa tämän esityksen sisällöstä on hyvin relevanttia myös
      noilla alustoilla -- pakkaus tapahtuu samoin, grafiikka luodaan
      samoilla menetelmillä, ja natiivit ohjelmat ovat tänä päivänä
      useimmiten AMD64-konekieltä.
    \item Ja sitten lienee syytä perustella, miksi itse lähdin
      tekemään kaikesta huolimatta Linux-alustalle.
  \end{itemize}
\end{frame}

\subsection{Miksi Linux?}
\begin{frame}{Mix just Linux?}
  \begin{itemize}
    \item Se nyt on lähellä sydäntä, tuo suomalaista alkuperää oleva
      avoimen lähdekoodin käyttöjärjestelmä\ldots
    \item Aiemmin tein pari 4k introa Linuxille, ja homma lähti
      luontevasti lapasesta, kun halusin kokea omin käsin, miten
      linukalla mentäisiin 1k-puolelle.
    \item Haaste on haaste. Linux-natiivi 4k oli jo sinänsä kova
      homma, joten tottakai piti kiivetä vielä korkeammalle.
    \item Viime vuodet olen myös luennoinut Jyväskylän yliopiston
      Informaatioteknologian tiedekunnan kurssia nimeltä
      ``Käyttöjärjestelmät''. Tässä harrastuksessa voi soveltaa mm.
      kyseisen kurssin sisältöä ja oppia siitä itse uutta.
    \item Eli yleisöstäkin ne, jotka ovat tulossa kevään 2016
      kurssille: itse uskoisin, että olette kurssilla oikein hyvissä
      käsissä, kiitos mm. tässä nähtävän pikku puuhastelun :).
    \item[$\rightarrow$] Vakuudeksi, tjsp., ajetaan käytännössä
      yksi pikkiriikkinen audiovisuaalinen ohjelma.
  \end{itemize}
\end{frame}

\section{Toteutus}
\subsection{Lähtökohtia}
\begin{frame}{Toteutus: Lähtökohtia}
\begin{itemize}
\item
Ei tosiaan keksitty koko pyörää uudelleen, vaan aiempia vinkkejä
tietysti noudateltiin.
\item Tällaiset olivat varmaan tärkeimpiä:
  \begin{itemize}
    \item Markku Reunanen (marq): ``Linux 4k coding'', Assembly 2006,
      \url{https://www.youtube.com/watch?v=-UQSiRg8Ra0}
    \item Brian Raiter: ``A Whirlwind Tutorial on Creating Really
      Teensy ELF Executables for Linux``, 
      \url{http://www.muppetlabs.com/~breadbox/software/tiny/teensy.html}
  \end{itemize}
\item
   Lisäksi on paljon keskusteltu kasvotusten muiden harrastajien
   kanssa.
\item
HUOM: Audiovisuaalisen tuotoksen tekemiseksi on käytännössä aivan
pakko hyödyntää apukirjastoja. Reunasen malliin käytän SDL:ää
(nykyversio SDL2) ja OpenGL:ää, jotka löytyvät useimmista
Linux-distroista oletuksena. Rohkenisin lisätä ``sallittujen
kirjastojen'' listaan myös perusasennuksista löytyvän OpenGL Utility
-kirjaston (libGLU.so). Ehkä muitakin?
\end{itemize}
\end{frame}

\subsection{Perustemput}
\begin{frame}{Toteutus: Perustemput}
  \begin{itemize}
    \item Se korkeintaan 1024 tavun mittainen Linux-ohjelma, joka
      kilpailupaikalla ajetaan, on itse asiassa ``kikka-kolmosella''
      rakenneltu mötkö, joka näyttäytyy shell-skriptinä, mutta itse
      asiassa avaa varsinaisen gzip-pakatun binäärin ``/tmp/''
      -hakemistoon (johon nykyisellä käyttäjällä on kirjoitus- ja
      suoritusoikeus) ja suorittaa sen sieltä. Kohteliaimmillaan
      väliaikainen tiedosto myös lopuksi poistetaan. Kilpailupaikan
      säännötkin voivat määrätä poistosta.
    \item[$\rightarrow$] katsotaan käytännössä, miltä tuo tiedosto
      näyttää.
    \item Niin\ldots pakattu data on käännetyssä järjestyksessä (eka
      tavu on tiedoston lopussa), koska tällä tapaa skriptiosuudesta
      saa yhtä tavua lyhyemmän. Nykyinen ``zcat'' -ohjelman toteutus
      ei hajoile lopussa tulevaan tauhkaan, joten nähtävästi näin voi
      tehdä.
    \item Reunasen esitelmästä modattu koodi on ihan standardi skripti
      (siihen asti, että skriptin sisälle onkin ängetty binääridataa,
      mikä on melkoisen ihanan törkeää :))
  \end{itemize}
\end{frame}

\subsection{Skripti ja gzip}
\begin{frame}{Toteutus: Rakentelu}
  \begin{itemize}
    \item Pelikenttä on siis: tehdään suoritettava binääritiedosto,
      pieni tai isompikin, kunhan se pakkautuu gzip-formaatissa
      (``deflate''-pakkaus) riittävän pieneksi.
    \item Zopfli -pakkausohjelma tekee tosi tiivistä tavaraa ja
      soveltuu nähtävästi hyvin myös pienten tiedostojen pakkaamiseen:
      \url{https://github.com/google/zopfli}
    \item[$\rightarrow$] katsotaan käytännössä, miten lopullinen
      tiedosto rakennellaan (ts. Makefile).
    \item Käynnistysohjelmassa uskallan korvata hyvin käyttäytyvän
      ``exit'' -komennon kolme tavua lyhyemmällä syntaksivirheellä eli
      yksinäisellä sulkumerkillä ``)'', koska loppuuhan se skriptin
      suoritus siihenkin\ldots en kylläkään ole toistaiseksi
      tarkistanut, miten yleispätevää tämä on.
    \item Loppuisi skripti myös binääritauhkan läpikäyntiin, mutta se
      olisi jo liian törkeää. Pahimmalla säkällä pakatussa datassa
      sattuisi olemaan jotain järkeviä komentoja, jotka sotkevat
      käyttäjän dataa!! Siihen ei mennä ollenkaan\ldots
  \end{itemize}
\end{frame}

\subsection{ELF-Binääri}
\begin{frame}{Toteutus: Pakattavan binääritiedoston muoto}
  \begin{itemize}
    \item Suoritettavan binääritiedoston on oltava ELF-formaatin
      mukainen siinä määrin, että nykyiset Linux-distrot noin
      keskimäärin hyväksyvät ja suostuvat käynnistämään sen,
      mahdollisista rikkeistä huolimatta.
    \item[$\rightarrow$] Katsotaan käytännössä, millaisen tiedoston
      käynnistysskripti purkaa ja käynnistää
    \item Suoritettavan tiedoston sisällössä on paljon luvallisia ja
      luvattomia mahdollisuuksia vaikuttaa siihen, miten hyvin se
      pakkautuu esim. zopfli:lla.
    \item Tässä produktiossa päädyin tekemään koko binäärin,
      mukaanlukien käyttöjärjestelmän vaatimat otsikkotiedot,
      NASM-assemblerilla tavu tavulta Raiterin ``teensy ELF'' -ohjeita
      mukaillen.
    \item[$\rightarrow$] katsotaan käytännössä, miltä 1k-tuotoksen
      varsinainen ohjelmakoodi näyttää Assemblerina tekstieditorissa.
  \end{itemize}
\end{frame}

\subsection{Superdoku?}
\begin{frame}{Toteutus: Sommittelutehtävä Sudokua siistimpi?}
  \begin{itemize}
    \item Deflate-pakkauksen teho perustuu (1) yleisimmin käytettyjen
      tavujen toistuvuuteen (2) yhtenäisten tavujonojen
      toistuvuuteen.
    \item Lopputulema on siis sitä pienempi, mitä enemmän pystyy
      käyttämään (1) kaikkein yleisimpiä tavuja/merkkejä ja (2) samoja
      yhtenäisiä tavujonoja useammassa kohdassa.
    \item Elintärkeätä on tietää, mitä tavuja ja tavujonoja voisi
      hyödyntää, ja toisaalta mitkä pätkät nykyisessä versiossa ovat
      pahinta ``myrkkyä'' pakkaukselle.
    \item Pari todella miellyttävää työkalua (varsin harmillisesti
      eivät ole avointa lähdekoodia, vaikkakin ilmaisia): ``defdb'' ja
      ``gzthermal''\\{\tiny
      \url{http://encode.ru/threads/1428-defdb-a-tool-to-dump-the-deflate-stream-from-gz-and-png-files}\\
      \url{http://encode.ru/threads/1889-gzthermal-pseudo-thermal-view-of-Gzip-Deflate-compression-efficiency}\\}
    \item[$\rightarrow$] katsotaan käytännössä työkaluohjelmien tulosteita.
  \end{itemize}
\end{frame}

\subsection{Törkeyksiä}
\begin{frame}{Toteutus: Törkeyksiä}
  \begin{itemize}
    \item ELF:n rakenteiden ja kenttien pituudet voi (osin) valehdella
      erilaisiksi kuin ne oikeasti ovat. Eli tietysti mahdollisimman
      samoiksi, että pakkautuu toisteisuuden vuoksi hyvin!
    \item Pirulainen näköjään tarkistaa kuitenkin mm., että
      merkkijonoksi väitetyt muistialueet päättyvät nollaan, joten
      aivan kaikki ``ei mene läpi'' kuitenkaan. Muitakin tarkistuksia
      on.
    \item[$\rightarrow$] Katsotaan näitä törkeyksiä käytännössä.
    \item Standardia rikkova binääri on luonnonvastainen hirviö, eikä
      se välttämättä toimi kaikissa järjestelmissä tai saman
      järjestelmän huomisessa versiossa.
    \item Kilpailujen luonteen mukaisesti 1024 tavun version tarvitsee
      kuitenkin toimia vain yhtenä päivänä ja vain sillä yhdellä
      koneella, jota kisapaikalla käytetään :)
    \item Olen yrittänyt pitää lähdekoodin sellaisena, että myös aivan
      standardin mukaan toimivan ohjelman voi tuottaa melko helposti,
      vaikkei se pakkaudukaan yhtä hyvin.
    \item 4k-sarjassa en välttämättä menisi näihin äärimmäisyyksiin
      alkuunkaan. Siellä on muutenkin 3072 tavua enemmän tilaa
      mellastaa.
  \end{itemize}
\end{frame}

\subsection{Kikkailua}
\begin{frame}{Toteutus: Muita kikkoja (outoja, ei törkeitä)}
%  Törkeyksien lisäksi aivan sallittuja keinoja ovat mm.:
  \begin{itemize}
  \item 32-bittisen (tai jopa 16-bittisen) käskykannan käyttäminen
    aina, kun mahdollista. Variaatiot saman vaikutuksen tuottavista
    käskyistä. Esim. PUSH \& POP vs. MOV.
  \item Kirjasto-ohjelmien linkittäminen siinä järjestyksessä kuin
    niitä tullaan kutsumaan. Kutsujen konekielikoodi toistuu silloin
    identtisenä, ml. ``kutsuosoite++''.
  \item Liukulukujen ``pyöristäminen'' siten, että esitysmuoto tavuina
    on melkein sama kuin jollain muulla koodilla tai
    datalla. Tuollaista mukavaa työkalua käyttelin:
    \url{http://www.h-schmidt.net/FloatConverter/IEEE754.html}
  \item GLSL-varjostinkoodin muuttujanimien vaihtaminen
    ASCII-merkeiksi, jotka ovat tiedostossa jo valmiiksi yleisiä.
  \item Viittaukset GLSL:n vektorielementteihin mahdollisimman
    yleisesti käytetyillä merkeillä (``xyzw'' vs. ``rgba''
    vs. ``stpq'').
  \item Ja muuta, ja muuta \ldots
  \item[$\rightarrow$] Katsotaan käytännössä
  \end{itemize}
\end{frame}


\subsection{Havaintoja}
\begin{frame}{Havaintoja 1/2}
  \begin{itemize}
  \item Tämän ekskursion perusteella Linuxilla ei taideta koskaan
    taistella 256 tavun sarjassa. En ainakaan itse näe mitään
    mahdollisuutta. Lienee käytännössä Java, Javascript tai MS-DOS,
    millä pystyy. Pienin ylimäärä lienee MS-DOSin COM-formaatissa(?).
  \item 1024 tavun sarjassa 64-bittinen Linux on mahdollinen, mutta
    luultavasti verrattain rajoittava alusta, johtuen ELFin ja
    funktiolinkitysten välttämättömästä tilantarpeesta.
  \item Tarkempaan vertailuun en pysty, kun en tunne muita
    nykyalustoja niin hyvin, mutta luulen, että niissä on vähemmän
    ylimääräistä pakollista. Windowsilla voi myös käyttää MIDIä
    musiikkiin, joten syntetisaattorikoodin tilalle mahtuu grafiikkaa.
  \item 32-bittinen rajoittaisi todennäköisesti paljon
    vähemmän. Mm. ELF-otsikot, tietyt merkkijonot, muistiosoitteet ja
    suurin osa konekielikäskyistä lyhempiä. Useammat Raiterin
    härökikoista käytettävissä. En ole vielä kokeillut.
    %Käytännössä koko koodi olisi erilainen\ldots
  \item 4096 tavun sarjassa pienempi ero, koska ylimäärän jälkeen
    jää kuitenkin selvästi yli kolme kiloa sisällölle.
  \end{itemize}
\end{frame}

\begin{frame}{Havaintoja 2/2}
  \begin{itemize}
  \item Ainakin 64-bittisessä Linux 1k:ssa on turha haaveilla
    kummemmasta softasyntikasta.
  \item Toisaalta rajoittaa, toisaalta vapauttaa: audiogenerointi on
    vahvasti osa uniikkia teosta.
  \item Grafiikan osalta sama homma. Kummempaa aliohjelmakirjastoa ei
    mukaan laiteta, vaan kaikki koodi tulee ``juuri tässä ja tätä
    teosta varten''. Ei tule rönsyjä.
  \item Jännittävä lisähavainto: Neljä kiloa ei jotenkin tunnu enää
    ollenkaan ahtaalta!
  \end{itemize}
\end{frame}

\section{Yhteenveto}
\begin{frame}{Kannattiko?}
  \begin{itemize}
  \item Relevantti kysymys lienee ``kannattiko taas lähteä''?
  \item Vahvasti kannatti. Miksi?
  \item Pelataan ensiksi harrastuskortti: Tykkään näpertää, ja
    näperrellä sain. Asetin haastavan tavoitteen ja pääsin
    siihen. Voittajafiilis. Jes. Sen takia kait harrastukset
    on\ldots
  \item Sitten nörttikortti: Tuli (jälleen kerran) luettua entistä
    tarkemmin ELF-spesifikaatio, AMD64:n manuaaleja (kohtalaisen
    laajalti), POSIX-standardin osia, Linuxin järjestelmärajapinnan
    osia. Kaikki on taas kerran vähän selkeämpää, mikä on aina
    miellyttävä kokemus.
  \item Työelämärelevanssikortti: Ensinnäkin voi taas viilata omaa
    kurssia (Käyttöjärjestelmät) tarkemmaksi ja hyödyllisemmäksi
    opiskelijoillekin. Assembleriin ja liukulukuihin liittyvät
    havainnot eivät ainakaan haittaa tutkimuskoodeja. Ohimennen näkee
    ja oppii kaikenlaista myös käsissä olevan kysymyksen ulkopuolelta.
  \item Omat kortit on siinä pöydällä, ja kyllähän tuo käsi aika
    vahvalta tuntuu.
  \end{itemize}
\end{frame}

\begin{frame}{Mitä voisi tehdä hurjemmin (eli ns. ``oikein'')?}
  \begin{itemize}
    \item Pakkauksen huomiointi ``minifier''-tyyppisissä ohjelmissa
      WebGL:lle ja Javascriptille. Näyttäisi olevan tutkittavaa ja
      kehitettävää -- mahdollisesti jopa liiketoimintapotentiaalia,
      koska kysyntää on varmasti kaikkialla, missä siirtokaistalla on
      rajat! (Itse näkisin mieluiten avoimen koodin toteutuksia ja
      liiketoiminnat palveluina koodia hyödyntäen, but that's just
      me\ldots)
    \item ``Pyynnöstä törkeän'' linkkerin luominen Linuxille, eli
      jotakuinkin Windows-puolelta tutun Crinklerin vastine, joka
      hoitaisi tässä käsipelillä kopeloidut muunnokset ELFiin
      automaattisesti. Olisi kiva koodata esim. C:llä ja antaa
      käännös- ja linkitysjärjestelmän hoitaa likaisten
      yksityiskohtien lisäksi myös törkeät. Varmasti ei
      liiketoimintapotentiaalia, mutta tällaisen värkin onnistuneesti
      tehnyt tyyppi pääsee kyllä töihin koodariksi. Välineellä voisi
      ehkä olla oikeampiakin käyttötarkoituksia.
    \item Pakkauksen huomiointi natiivin AMD64-konekielen
      optimoinnissa. Sama keissi -- tuskin bisnestä, mutta älytön
      taidonnäyte portfoliossa ja mieletön osaamisen määrä matkan
      varrelta.
  \end{itemize}
\end{frame}

\begin{frame}{Inception:}
  \begin{itemize}
    \item Mitä ajatuksia ehkä haluaisin istuttaa erityisesti JY:n
      tietotekniikan opiskelijoiden päähän?
    \item Vaikka tämä on harrastelua, niin kyllähän hyvä demokoodari
      on hyvä koodari muutenkin.
    \item Syvään päähän sukeltaminen voi olla tuskallista, mutta
      sieltä voi myös tulla palkitsevia löytöjä.
    \item Esim. edellisen kalvon työkalut, hyvin tehtynä,
      soveltuisivat mielestäni hyvin esim. kanditutkielman ja sitä
      seuraavan gradun ja ehkä erikoistyönkin aiheiksi.
    \item Pohjalle suosiolla kurssit Ohjelmointi 1\&2, Algoritmit
      1\&2, Käyttöjärjestelmät, Automaatit ja kieliopit,
      Funktio-ohjelmointi 1\&2 (laajoina versioina),
      Tietokonegrafiikan perusteet \& Reaaliaikainen renderöinti
      (GLSL:n osalta), Web-sovellukset (Javascriptin osalta),
      Ohjelmointikielten periaatteet ja Kääntäjätekniikka.
    \item Kandi pakkausalgoritmeista ja/tai ohjelmointikielten
      jäsentämisestä, sitten gradu minifiereista / optimoinneista;
      konkreettinen koodi todisteeksi, että näin se tehdään.
    \item[] Ois aika kova.
    \item[] (toki alustava ajatus; vaatii tarkempaa aiheen
      pallottelua)
  \end{itemize}
\end{frame}

\begin{frame}{Se oli siinä tällä kertaa}
  \begin{itemize}
    \item Keskustelua?
  \end{itemize}
\end{frame}
\end{document}
