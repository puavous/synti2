% -*- coding: utf-8 -*-

\documentclass[pdf,10pt]{beamer}
\usepackage{amsfonts}
\usepackage{amsmath}

\mode<presentation>
{
  \usetheme{Goettingen}
  \usecolortheme{seagull}
}

\setbeamertemplate{navigation symbols}{}
\setbeamertemplate{bibliography item}{[\theenumiv]}

\setlength{\parskip}{\baselineskip}


\usepackage[finnish]{babel}
\usepackage[utf8]{inputenc}

\usepackage{palatino}

%\graphicspath{{figs/}}

\title[TIEP115]
{Tietotekniikasta, matematiikasta, harrastamisesta ja soitinrakennuksesta}

\author{Paavo Nieminen}

\date{TIEP115 Johdatus tietotekniikkaan, 12.2.2015}

\begin{document}

\begin{frame}
  \titlepage
\end{frame}

\begin{frame}{Sisältö}

{
\setlength{\parskip}{\smallskipamount}
  \tableofcontents
  % You might wish to add the option [pausesections]
}
\end{frame}

\begin{frame}{Tässä esitelmässä:}
  \begin{itemize}
  \item Lähdetään liikkeelle ''viattomasta ja hauskasta harrastusprojektista''
  \item Katsotaan, mihin tietotekniikan osa-alueeseen se liittyy
  \item Millaista historiaa tuolla tietotekniikan osa-alueella on
  \item Mitä aiheen tutkimiseen liittyy (muun muassa)
  \item Miten laitoksellamme tarjottavat kandidaatin ja maisterin
    opinnot liittyvät aiheeseen.
  \item Väliajoin totean erikseen ''pointteja'', joita haluan korostaa
    erityisesti ensimmäisen vuoden opiskelijalle.
  \end{itemize}
\end{frame}


\section{Synti2}
\begin{frame}{''Synti2 Software Synthesizer''}
\begin{itemize}
  \item Ohjelmisto, joka muodostaa ääntä algoritmisesti,
    ''softasyntikka''
  \item Harrastusprojekti, johon tulee kajottua kerran tai pari vuodessa
  \item Julkinen versiokanta:\\ \small{\url{https://yousource.it.jyu.fi/synti2/synti2}}

    (pittäis löytyä myös googlettamalla ''synti2'')

  \item Kehitysympäristönä tarvitaan ainakin toistaiseksi Linux ja
    useampia lisäkirjastoja sekä työkaluohjelmia.

\end{itemize}
\end{frame}

\begin{frame}{''Synti2 Software Synthesizer'' --~synteesimenetelmä}

  \begin{itemize}
  \item Vaihemodulaatiosynteesi (tyyliin Yamahan DX7 ym. 80-luvulta)
  \item Operaattorit miksattavissa (ts. osin additiivinen synteesi)
  \item Resonoiva ali- / yli- / kaistapäästösuodin (ts. osin
    subtraktiivinen synteesi)
  \item Oskillattorit esilaskettuina taulukoina
    (''wavetable-synteesi'')
  \item Rajalliset viivästysefektit / kampasuodin mahdollisia
  \end{itemize}
\end{frame}

\begin{frame}{''Synti2 Software Synthesizer'' --~ominaisuudet}
  \begin{itemize}
  \item Reaaliaikainen äänentuotto 
  \item Ohjaus standardilla MIDI-protokollalla
  \item Tyypillisen kosketinsoittimen perusominaisuudet, ts. huomioi
    kosketusvoimakkuuden sekä jatkuva-arvoiset kontrollerit (ml. pitch
    bend, modulator, aftertouch, breath, \ldots)
  \item Hetkellinen tila luettavissa esim. visualisointia varten
  \item Sekvensserimoodi nauhoitetun kappaleen toistoon
  \item Yksinkertainen sekvenssi, äänipankki ja toistokoneisto mahtuu
    3 kilotavuun pakattua suoritettavaa koodia.
  \end{itemize}
\end{frame}

\begin{frame}{Miksi, miksi, miksi?}
  Miksi näitä juttuja pitäisi vielä harrastaakin? Eikö se riitä, että
  näitä tekee töikseen tai opiskellakseen?
  \begin{itemize}
    \item Harrastuksissa oppii asioita, jotka eivät (ainakaa
      vielä\ldots) tule vastaan opinnoissa tai töissä.
    \item Oppi ei kaada ojaan koskaan. Enemmän se avaa ovia kuin
      sulkee.
    \item Harrastukset ovat hauskoja ja antavat motivaatiota teorian
      oppimiseen yliopistokursseilla, joissa ei välttämättä ehditä
      sovelluksiin asti.
    \item Voipi välillä elvistellä tekeleillään vähän.
    \item Optimitilanteessa harrastuksesta voi tulla työ tai siitä voi
      lähteä liikkeelle myyntikelpoinen tuote\ldots
  \end{itemize}
\end{frame}

\begin{frame}{Pointti: Harrastakaa vapaasti!}
  On sopivaa ja suotavaa kokeilla, löytyisikö opintojen lähimaastosta
  aihetta myös harrastelulle. Itseäni kiinnostaa äänisynteesi ja
  audio-ohjelmistot, mutta yhtä hyvin kiinnostuksen kohteet voivat
  olla grafiikassa, peleissä, tietoturvassa, verkkokaupoissa,
  sosiaalisessa mediassa, kääntäjätekniikassa, \ldots

  Opintoaika on rajoitettu, joten kiirettä pitelee kurssienkin
  parissa. Silti aika ennen työelämää mahdollistaa harrastukset
  paremmin. Sopivasti valittu harrastus voi myös konvergoida opintojen
  kanssa etenkin loppua kohden edettäessä.

  Synti2:n parissa olen itse oppinut asian jos toisenkin
  mm. ohjelmoinnista, käyttöjärjestelmistä ja tietokonegrafiikasta.
  Päätarkoitus ja alullepaneva tekijä oli kuitenkin opetella
  nimenomaan digitaalista soitinrakennusta, joten puhutaan tällä
  luennolla siitä.
\end{frame}

\section{Historiaa}

\subsection{Tietokonemusiikki}
\begin{frame}{Historiaa: tietokonemusiikki, digitaalinen äänisynteesi}
  Digitaalisen tietokoneen käyttöä ''musiikkisoittimena'' visioidaan
  1960-luvulla esim. Mathewsin artikkelissa
  \cite{Mathews63thedigital}, jossa viitataan jo ensimmäisiin olemassa
  olleisiin toteutuksiin.

  Törmäsin Mathewsin artikkeliin Smithin konferenssipuheen
  \cite{smith1991viewpoints} alkulauseessa. Itse puhe, tai pikemminkin siihen
  liittyvä artikkeli, on sinällään mielenkiintoista luettavaa, vaikka
  onkin esseemäinen näkemys historiaan ja siihen, mikä kirjoittajan
  mielestä onnistui ja mikä meni pieleen 1960-luvun jälkeen. Vuonna
  1991 kirjoitetun esseen näkemyksiä voidaan nyt jo arvioida
  historiallisesta näkökulmasta. Tietokoneiden nopeuden kasvu on kauan
  sitten mahdollistanut tarvittavan laskennan, mutta äänten ja
  musiikin kuvailemiseen tarkoitettujen kielten (eli
  musiikin ''ohjelmointikielten'') osalta lopullisten voittojen
  saavuttaminen voi olla vaikeampaa, kuten Smith alkusanoissaan
  ounastelikin.
  
\end{frame}

\subsection{Äänisynteesi}
\begin{frame}{Erilaisia digitaalisen äänisynteesin menetelmiä}
  Synti2:n synteesimenetelmä perustuu (tavallaan) taajuusmodulaatioon,
  jonka sovellukset tietoliikenteessä olivat arkipäivää jo siinä
  vaiheessa, kun niiden käyttöä musiikilliseen äänisynteesiin vasta
  ehdotettiin 1970-luvulla \cite{chowning1973fm}.

  Muita viime vuosituhannen loppuun mennessä keksittyjä
  äänisynteesimenetelmiä on lueteltu ja luokiteltu esimerkiksi Tolosen
  ym. raportissa \cite{TolonenEtal1998evaluation}. Tulevaisuudessa,
  tietokoneiden suorituskyvyn kasvaessa, fysikaaliseen mallinnukseen
  perustuvien menetelmien roolin voisi olettaa yhä kasvavan.

\end{frame}

\section{DSP}
\begin{frame}{Historiaa: Digitaalinen signaalinkäsittely}

  Digitaalinen äänisynteesi on yksi sovellus ja osa-alue
  laajemmasta tutkimuskentästä nimeltä digitaalinen signaalinkäsittely
  (engl. digital signal processing, DSP), jossa yleisesti ottaen
  vieläkin laajemman alan, signaalinkäsittelyn, ongelmia pyritään
  ratkaisemaan digitaalitekniikan, ml. tietokoneohjelmien, avulla.

  Yhtenä DSP-alan kivijalkana on mm. Shannonin informaation siirron
  matemaattisia perusteita koskeva julkaisu
  \cite{shannon48amathematical} vuodelta 1948.

  Muita alan sovelluksia kuin musiikki-instrumentit? Puhelinyhteydet,
  televisio- ja radiolähetykset, tallennusvälineet, mittarit autojen
  moottoreissa, lääketieteellinen kuvantaminen, eksoplaneettojen
  etsiminen, \ldots
  
\end{frame}

\subsection{Suotimet}
\begin{frame}

  Esimerkki digitaaliseen signaalinkäsittelyn kenttään liittyvästä
  erityiskohteesta on suodinsuunnittelu (filter design).

  Synti2:n suodin (on tietysti vaan copy-pastattu nettifoorumilta,
  mutta jos haluaisi tehdä aiheesta tiedettä, niin tulisi tietää että
  se\ldots) perustuu Dattorron artikkeliin \cite{dattorro1997effect},
  joka vie eteenpäin 1980-luvulla julkaistun kirjan
  \cite{Chamberlin1980} ideaa.

\end{frame}

\subsection{FFT}

\begin{frame}
  Tärkeä ominaisuus esimerkiksi musiikissa on että suotimien ja muiden
  synteesimenetelmän osien tuottaman äänen ns. harmoninen sisältö eli
  taajuusalueen spektri saadaan sopivaksi. Taajuuksia päästään
  tutkimaan (ja tarvittaessa manipuloimaan) muuntamalla ajan suhteen
  näytteistetty digitaalinen signaali esimerkiksi diskreetillä
  Fourier-muunnoksella (engl. discrete Fourier transform,
  DFT). Matemaattisesti sanoen: signaalia kuvaava vektori siirretään
  toiseen avaruuteen, jonka kanta on helpommin tulkittavissa
  tarkoituksenmukaisella tavalla, eli DFT:n tapauksessa
  taajuuskomponentteina.
\end{frame}

\begin{frame}
  Cooleyn ja Tukeyn julkaisu \cite{CooleyTukey1965fft} teki tunnetuksi
  nopean menetelmän diskreettien signaalien taajuuskomponenttien
  analysoimiseksi, ns. nopea Fourier-muunnos (fast Fourier transform)
  FFT. Algoritmin historian seuraaminen johtaa kuitenkin jo Carl
  Friedrich Gaussin tuntemiin menettelyihin
  \footnote{\url{http://music.columbia.edu/cmc/musicandcomputers/popups/chapter3/xbit_3_2.php}}.
\end{frame}


\begin{frame}{Pointti: Keskittykää kursseihin alusta asti!}
  Hyödyllisen algoritmin vaativuusluokan pudottaminen luokasta
  $O(N^2)$ luokkaan $O(N \mathop{log} N)$ takaa tietoteknikolle
  ikuisen paikan historian kirjoissa! Mitä edellinen lause tarkoittaa,
  selviää kursseilla Algoritmit 1 \& Algoritmit 2.

  Pitäkää mielessä sovellukset ja tavoitteet jo alkupään opinnoissa,
  vaikka teoriapainotteisilla kursseilla ei niitä kaikkia ehditä
  esittelemään. Ei alkupään teoria edes kaikkia sovelluksia
  mahdollista -- mutta pohja tulevalle luodaan!
  
\end{frame}

\begin{frame}{Yleinen teoria mahdollistaa uudet tulokset}
  Erilaiset wavelet-muunnokset \cite{daubechies1990wavelet} ovat
  Fourier-muunnoksen ohella toinen, tietyissä tilanteissa
  tarkoituksenmukainen tapa siirtää diskreetti signaali avaruuteen,
  jonka kanta on alkuperäistä helpompi. Reunahuomautus: yhtymäkohtana
  aiempaan aiheeseen, Wavelet-muunnos voidaan tulkita tietyllä tapaa
  järjestelmällisesti suunniteltujen yli- ja alipäästösuotimien sovelluksena.

  Kaikkien näiden asioiden äärelle päästään matematiikan
  ''peruskursseilla'' eli noin kolmannen yliopistovuoden syventävillä
  kursseilla, kun niitä on ensin ehditty pohjustaa perustiedoilla
  esimerkiksi vektoriavaruuksista ja niiden kannoista sekä
  fundamentaaleilla funktioiden ja raja-arvojen käsitteillä.

\end{frame}

\begin{frame}{Pointti: Matematiikasta on hyväksyttävää innostua!}

  Kanditutkintoon mahtuu helposti ainakin matematiikan perusopinnot
  (25op eli melkein 700 tuntia aikaa opetella matematiikan eri
  osa-alueita.)

  Käyttäkää hyvät ihmiset hyödyksi tuo aika, joka on sitä varten nyt
  annettu\ldots työelämässä voi sitten päivittää ja ylläpitää tietoja,
  mutta perustaitoja ei enää välttämättä ehdi siinä vaiheessa haalia.

  Mikäli esimerkiksi äänen- ja kuvankäsittelyn teknologiat alkavat
  kiinnostaa, ei ole paha ottaa jopa perus- ja aineopintojen
  pakettia (60 op on jo yli 1600 tuntia ihanaa matematiikka-aikaa!)

\end{frame}



\section{Nykypäivää}
\begin{frame}{Nykypäivää}

  Tutkimuksen laajuus tänään? Hakusanalla ''sound synthesis'' löytyy
  Thomson Reuters Web of Knowledgesta vuoden 2010 jälkeen julkaistuja
  lehtiartikkeleita 125 kpl. Esimerkiksi ehdotus kvanttihiukkasen
  käyttäytymiseen perustuvasta synteesimenetelmästä
  \cite{CadizRamos2014quantum}.

  TODO: Entä hakusanalla ''filter design'' vuoden 2010 jälkeen?

  Uudempi paperi suodintekniikasta musiikkisoitinten tarpeisiin on
  esim. \cite{Wishnick14timevaryingfilters}.

  TODO: Entä hakusanalla ''digital signal processing'' vuoden 2010 jälkeen?

\end{frame}


\section{Lähteitä}

\begin{frame}{Pointti: opetelkaa hakemaan tietoa laajasti (mutta hallitusti)}
  Papereita julkaistaan nykyään tosi paljon, ja niitä on helppo
  löytää. Perusasiat on kaluttu kauan sitten ja nyt ollaan taas jo
  uuden äärellä. On kriittisen tärkeätä kyetä kartoittamaan ja
  valitsemaan relevantit julkaisut, jotka liittyvät omaan
  tiedontarpeeseen. Kannattaa alkaa opettelemaan jo hyvissä ajoin
  ennen tutkielmien tekoa. Itseä kiinnostava aihekin löytyy helpommin,
  kun on jo selaillut, mitä muut ovat aiemmin kirjoittaneet aihepiirin
  ympäriltä.

  Tietoa tulee niin paljon ja monesta tuutista, että esimerkiksi
  ensimmäiset Google-osumat saattavat olla yhtä tyhjän
  kanssa. Lähtökohtahan niistä voi hyvinkin löytyä\ldots Tätä luentoa
  varten hain ensimmäiseksi ''History of digital signal processing'',
  mutta syvemmältä piti kaivaa, ennen kuin seuraavien kalvojen
  lähdeluettelo lopulta löytyi. 
\end{frame}

\begin{frame}[allowframebreaks]{Lähteitä}
\tiny
\bibliographystyle{plain}
\bibliography{dspstuff}
\end{frame}


\end{document}


\begin{frame}{Softasyntikan toimintaympäristö}
  \begin{itemize}
  \item DA-muunnin, digitaalisen datan puskuri äänikortissa,
    ajuriohjelmistot, käyttöjärjestelmä
  \item Kirjasto-ohjelmat järjestelmän abstrahoimiseksi
  \item Ohjelmoijan rajapinta: aika-ikkuna täytettäväksi numeroilla,
    esim. 256 tavua, joista ulostulosignaalin amplitudi määräytyy
    esim. 48000 kertaa sekunnissa kanavaa kohti. Ikkuna on täytettävä
    salamannopeasti, jotta ''ääni ei pätki''.
  \item ''Musiikkia'' varten pitää jostain tulla myös tieto rytmistä,
    äänenkorkeudesta ja -väristä sekä harmoniasta. Eli
    ''instrumentit'' ja ''nuottien sekvenssi'' ilmeisesti tarvitaan.
  \item Eli tavoite on hyvin selkeä. Sitte eiku koodaamaan.
  \end{itemize}
\end{frame}

\section{Ekat purkat: Synti}
\begin{frame}{tästä se lähti}
\begin{itemize}
\item Teinistä asti haikaillut koodailla näitä, mutta ei tullut
  aloitettua\ldots
\item Innostukseni heräsi uudelleen Zipolan 4k introsta ekoilla
  Instansseilla pari vuotta sitten. ''Mäkin haluun tehdä meteliä!''
\item Mitäs meillä pittäis olla? Varmaan fraasisekvensseri, jotain
  oskillaattoreita, verhokäyriä, filttereitä, \ldots ja kiva nimi
  projektille.
\item Niinpä ''tein Syntiä'' pari kuukautta ennen Instanssia 2011.
\item Syntyi ekat purkkaviritykset \ldots
\item Hyvät tuli.
\item Meteliä! Jes!!!
\end{itemize}
\end{frame}


\begin{frame}{Mutta..}
\begin{itemize}
\item
  Musiikkia hankala editoida (tekstimuotoinen ``oma formaatti''\ldots
  tyhmä idea). Ajattelin säveltää ekan biisin jälkeen myös toisen,
  mutta kunto loppui syötedatan muotoiluun ekan bassoriffin kohdalla.
\item
  Melko pikkiriikkistä koodia, mutta ei kovin nopeata -- ei
  reaaliaikaisesti soiteltavissa.
\item
  Kuvan synkronointia äänen kanssa ei ollut tullut ajateltua yhtään
  alunperin -- vain purkka ja päälleliimaus olivat mahdollisia
  varsinaisen 4k intron tekemisen hetkellä.
\item
  Sounditkin oli vähän niin ja näin, vaikka vahingossa toteutuneella
  ''köyhän miehen taajuusmodulaatiolla'' sai ihan riittävän pahvisen
  kumivirvelin.
\item
  Koodissa jopa niin hämmästyttävän paljon purkkaa, ettei kehdannut
  julkaista.
\end{itemize}
\end{frame}


\begin{frame}{Kantapääopetuksia}
\begin{itemize}
\item
  Numero 1: Älä keksi MIDIä uudestaan; se on jo ihan hyvä
  (lähtökohdaksi, ei lopputulemaksi).
\item
  Siis kertaus: älä keksi uudestaan protokollaa, jonka
  insinööriarmeija on jo miettinyt läpi.
\item
  Numero 2: Valmiit työkalut (esim. sekvensseriohjelmat) ovat ihan
  hyviä ja niitä on mietitty. Hyödynnä!
\item
  Numero 3: Syntikka on kivampi reaaliaikaisena.
\item
  Numero 4: Koodinnälkä kasvaa koodatessa, jos innostuu jostain
  puuhasta.
\end{itemize}
\end{frame}


\section{Uudet purkat: Synti2}
\begin{frame}{Synti2:n vaatimukset 1/2}
\begin{itemize}
\item Kalusto tuottaa pikkiriikkisen suoritettavan ohjelman, jossa on
  mukana ääni ja grafiikka (so. ennakkoon kiinnitetty musiikillinen ja
  graafinen sekvenssi sekä syntesoidut
  instrumenttiäänet). Suoritettava ohjelma datoineen on kooltaan alle
  4096 tavua, ja sen voi suorittaa standardissa käyttöympäristössä
  (kehitysvaiheessa 32-bittinen Linux, libSDL, libGL, gunzip TAI
  64-bittinen Linux johon asennettuna 32-bittiset SDL \& OpenGL)
\item
  Grafiikan parametrit voidaan lukea syntetisaattorin tilasta, jota
  sekvenssi ohjaa. (Äänitapahtumiin synkronoitu animaatio)
\item
  Sävellysvaiheessa synteesi on reaaliaikainen. Latenssin on oltava
  alle 10 millisekuntia (jota suuremmat latenssit häiritsevät jopa
  meikäläistä rähmätassua sunnuntaisoittelijaa)
\end{itemize}
\end{frame}

\begin{frame}{Synti2:n vaatimukset 2/2}
\begin{itemize}
\item Sävellysvaiheessa ohjattavissa standardilla
  MIDIllä. Mahdollistaa musiikin (ja tarkemmin ajatellen myös
  animaation ohjaussekvenssin) tuottamisen millä tahansa
  MIDI-sekvensseriohjelmistolla.
\item Suoritettavan ohjelman sekvenssi muodostetaan (karsien ja
  muuntaen) standardimuotoisesta MIDI-musiikkitiedostosta (SMF0 tai
  SMF1).
\item Instrumentteja varten on parametrieditori, jossa jokainen
  parametri on muutettavissa hiirellä raahaamalla. Äänien tallennus ja
  lataus tapahtuu MIDI-maailmassa tavanomaisella SysEx-tiedostojen
  käytänteellä.
\item Äänenlaadultaan vähintään yhtä hyvä kuin alkuperäinen ``Synti
  ykkönen''
\item Lisävaatimus Instanssin aikataulun kirjoittajalta:
  synteesimoduulin koko on alle 1 kilotavu. Tämä oli ainoa, joka ei
  aivan täysin toteutunut :).
\end{itemize}
\end{frame}

\begin{frame}{Toteutuksesta}
\begin{itemize}
\item
  FM-synteesi, 4 siniaalto-operaattoria
\item
  Wavetablet, Verhokäyrät, Filtterit ja Delayt kuten edellisessä,
  joskin aavistuksen verran paremmin
\item 
  Sekvensseri
\item 
  MIDI-rajapinta
\item
  Apuohjelmat: reaaliaikainen MIDI-adapteri, offline
  SMF-adapteri, soundieditori.
\item
  Huhhuh!! Aika paljon sitten putkahtikin tekemistä pikkiriikkisen
  oskillaattorin ympärille \ldots
\end{itemize}
\end{frame}


\begin{frame}{Tilanne 9.3.2012 klo 20:38}
  \begin{itemize}
  \item
    Aika kiva lelu \ldots
  \item
    Katsotaanpa kalustoa livenä \ldots Demoefektin todennäköisyys oli
    kuin olikin suuri (\ldots yllättäen oli kuitenkin muu kuin oma
    softa, joka jumitti audiosysteemin.)
    \begin{itemize}
    \item Jack käyntiin
    \item Syntikka käyntiin
    \item ''MIDI-mukiloija'' käyntiin
    \item Äänieditori käyntiin
    \item Koskettimet kiinni ja jamittelemaan \ldots
    \end{itemize}
  \item Biisin säveltäminen vapaavalintaisella
    MIDI-sekvensserillä. Esim. Rosegarden on oikeasti hyvä ohjelma,
    vaikka oli huteralla tuulella tänään.
  \item Paketointi ''4k introksi'' on helppoa kuin 'make tinyexe'.
  \item Eli miten kävi otsikon lupaukselle ''alle 1k softasyntikka''?
    Synamoduulin saa runtattua 1.6 kiloon, mutta äänen saaminen
    äänikortista ulos asti (SDL:n kautta) pullahtaa minimissään 2
    kiloon.
  \end{itemize}
\end{frame}

\begin{frame}{Kantapääopetuksia}
\begin{itemize}
\item Vois lukea matematiikkaa silloin, kun siihen on oikeasti aikaa,
  vaikkapa perusopintojen yhteydessä. Signaaliprosessoinnin teoria
  (suodatinsuunnittelu ym.) helpottaisi näitä harrastuksia kovasti.
\item
  Vois koodata kaikenlaista useammin, ettei taidot ruostu esim. olio-,
  grafiikka- tai käyttöliittymäohjelmoinnin osalta.
\item
  Vois vaikka tämän jälkeen päästää irti menneisyydestä ja siirtyä
  muihin kuin C-kieleen.
\end{itemize}
\end{frame}

\begin{frame}{Lähdekoodit ja Workshoppailu Instansseilla}
  \begin{itemize}
  \item Lähdekoodit MIT-lisenssillä noin puoleen yöhön mennessä
    (Malttia!): \url{https://yousource.it.jyu.fi/synti2}
  \item Apuohjelmat vielä bugisia ja muutenkin hiukan
    vaiheessa. Kivoja ovat ikuinen silmukka reaaliaikasäikeessä tai
    täysillä korvaan tuleva valkoinen kohina. Varokaa, jos kokeilette.
  \item Parantunevat (koodit, ei korvat), jahka Instanssin kompojen
    deadlinet lähestyvät :)
  \item Nykikää hihoista, jos aihepiiri(t) kiinnostaa enempi.
  \end{itemize}
\end{frame}

\end{document}


