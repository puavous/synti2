% -*- coding: utf-8 -*-

\documentclass[pdf,10pt,handout]{beamer}
\usepackage{amsfonts}
\usepackage{amsmath}

\mode<presentation>
{
  \usetheme{Goettingen}
  \usecolortheme{seagull}
}

\setbeamertemplate{navigation symbols}{}
\setbeamertemplate{bibliography item}{[\theenumiv]}

\setlength{\parskip}{\baselineskip}


\usepackage[finnish]{babel}
\usepackage[utf8]{inputenc}

\usepackage{palatino}

%\graphicspath{{figs/}}

\title[TIEP115]
{Tietotekniikasta, harrastamisesta, matematiikasta ja soitinrakennuksesta}

\author{Paavo Nieminen}

\date{TIEP115 Johdatus tietotekniikkaan, 18.2.2016}

\begin{document}

\begin{frame}
  \titlepage
\end{frame}

\begin{frame}{Sisältö (sama kuin vuosi sitten)}

{
\setlength{\parskip}{\smallskipamount}
  \tableofcontents
  % You might wish to add the option [pausesections]
}
\end{frame}
\beamerdefaultoverlayspecification{<+->}

\begin{frame}{Tässä esityksessä:}
  \begin{itemize}
  \item Lähdetään liikkeelle tohtorikoulutettavan ''viattomasta ja
    hauskasta harrastusprojektista''
  \item Katsotaan, mihin tietotekniikan osa-alueeseen se liittyy
  \item Millaista historiaa tuolla tietotekniikan osa-alueella on
  \item Mitä aiheen tutkimiseen liittyy (muun muassa)
  \item Miten laitoksellamme tarjottavat kandidaatin ja maisterin
    opinnot liittyvät aiheeseen.
  \item Väliajoin totean erikseen ''pointteja'', joita haluan korostaa
    erityisesti ensimmäisen vuoden opiskelijalle.
  \end{itemize}
\end{frame}


\section{Harrastelua ja leikkimistä}
\begin{frame}{''Synti2 Software Synthesizer''}
\begin{itemize}
  \item Synti2 on, muusikkojargonilla ilmaistuna, mun omatekemä
    softasyntikka, jolla pääsee silloin tällöin elvistelemään. (Kiitos
    Jonnelle tämänkin kevään mahdollisuudesta!)
  \item Muusikkojargon avattuna tarkoittaa tietokoneohjelmaa
    (software, ''softa''), joka muodostaa ääntä (synthesizer,
    ''syntikka'')
  \item Harrastusprojekti, johon tulee kajottua kerran tai pari vuodessa
  \item Julkinen versiokanta:\\ \small{\url{https://yousource.it.jyu.fi/synti2/synti2}}

    (pittäis löytyä myös googlettamalla ''synti2'')

  \item Kehitysympäristönä tarvitaan ainakin toistaiseksi Linux ja
    useampia lisäkirjastoja sekä työkaluohjelmia.
    
  \item \ldots [Demotaan käytännössä, jos salitekniikka antaa myöten]
    \ldots

\end{itemize}
\end{frame}

\begin{frame}{''Synti2 Software Synthesizer'' --~synteesimenetelmä}

  \begin{itemize}
  \item Vaihemodulaatiosynteesi (tyyliin Yamahan DX7 ym. 80-luvulta)
  \item Operaattorit miksattavissa (ts. osin additiivinen synteesi;
    tyypillistä FM-syntikoissa)
  \item Resonoiva ali- / yli- / kaistapäästösuodin (ts. osin
    subtraktiivinen synteesi)
  \item Oskillattorit esilaskettuina taulukoina
    (''wavetable''-synteesi)
  \item Rajalliset viivästysefektit / kampasuodin mahdollisia
  \end{itemize}
\end{frame}

\begin{frame}{''Synti2 Software Synthesizer'' --~ominaisuudet}
  \begin{itemize}
  \item Reaaliaikainen äänentuotto 
  \item Ohjaus standardilla MIDI-protokollalla
  \item Sisältää tyypillisen kosketinsoittimen perusominaisuudet,
    ts. huomioi kosketusvoimakkuuden sekä jatkuva-arvoiset
    kontrollerit (ml. pitch bend, modulator, aftertouch, breath
    controller, \ldots)
  \item Hetkellinen tila luettavissa esim. musiikkiin synkronoitua
    visualisointia varten
  \item[] (ei tarvitse taputtaa)
  \item Sekvensserimoodi nauhoitetun kappaleen toistoon
  \item Yksinkertainen sekvenssi, äänipankki ja toistokoneisto mahtuu
    3 kilotavuun pakattua suoritettavaa koodia.
  \item[] (nyt saa taputtaa.)
  \end{itemize}
\end{frame}

\begin{frame}{Miksi, miksi, miksi?}
  Miksi näitä juttuja pitäisi vielä harrastaakin? Eikö se riitä, että
  näitä tekee töikseen tai opiskellakseen?
  \begin{itemize}
    \item Harrastuksissa oppii asioita, jotka eivät tule vastaan
      opinnoissa tai töissä (kunnes saattavat yllättäen tullakin).
    \item Oppi ei kaada ojaan koskaan. Enemmän se avaa ovia kuin
      sulkee.
    \item Harrastukset ovat hauskoja ja antavat motivaatiota teorian
      oppimiseen yliopistokursseilla, joissa ei välttämättä ehditä
      sovelluksiin asti.
    \item Voipi välillä elvistellä tekeleillään vähän.
    \item Optimitilanteessa harrastuksesta voi tulla työ tai siitä voi
      lähteä liikkeelle myyntikelpoinen tuote\ldots no, ainakin voi
      haaveilla \ldots
  \end{itemize}
\end{frame}

\begin{frame}{Pointti: Harrastakaa vapaasti!}
  \begin{itemize}
    \item
  On sopivaa ja suotavaa kokeilla, löytyisikö opintojen lähimaastosta
  aihetta myös harrastelulle. Itseäni kiinnostaa äänisynteesi ja
  audio-ohjelmistot, mutta yhtä hyvin kiinnostuksen kohteet voivat
  olla grafiikassa, peleissä, tietoturvassa, verkkokaupoissa,
  sosiaalisessa mediassa, kääntäjätekniikassa, tekoälyssä, \ldots

\item
  Opintoaika on rajoitettu, joten kiirettä pitelee kurssienkin
  parissa. Silti aika ennen työelämää mahdollistaa harrastukset
  paremmin. Sopivasti valittu harrastus voi myös konvergoida opintojen
  kanssa etenkin loppua kohden. Siis opintopisteitä ja tutkimusaihe
  harrastuksen tiimoilta? Täysin mahdollista!

\item
  Synti2:n parissa olen itse oppinut asian jos toisenkin
  mm. ohjelmoinnista, käyttöjärjestelmistä ja tietokonegrafiikasta.
  Päätarkoitus ja alullepaneva tekijä oli kuitenkin opetella
  nimenomaan ''digitaalista soitinrakennusta'', joten puhutaan tällä
  luennolla siitä.
\end{itemize}
\end{frame}

\section{Historiaa}

\subsection{Tietokonemusiikki}
\begin{frame}{Historiaa: tietokonemusiikki, digitaalinen äänisynteesi}
  \begin{itemize}
    \item
  Digitaalisen tietokoneen käyttöä ''musiikkisoittimena'' visioidaan
  1960-luvulla esim. Mathewsin artikkelissa
  \cite{Mathews63thedigital}, jossa jo viitataan ensimmäisiin olemassa
  olleisiin toteutuksiin.

    \item
  Törmäsin Mathewsin artikkeliin Smithin konferenssipuheen
  \cite{smith1991viewpoints} alkulauseessa. Itse puhe, tai pikemminkin siihen
  liittyvä artikkeli, on sinällään mielenkiintoista luettavaa, vaikka
  onkin esseemäinen näkemys historiaan ja siihen, mikä kirjoittajan
  mielestä onnistui ja mikä meni pieleen 1960-luvun jälkeen. Vuonna
  1991 kirjoitetun esseen näkemyksiä voidaan nyt jo arvioida
  historiallisesta näkökulmasta. Tietokoneiden nopeuden kasvu on kauan
  sitten mahdollistanut tarvittavan laskennan, mutta äänten ja
  musiikin kuvailemiseen tarkoitettujen kielten (eli
  musiikin ''ohjelmointikielten'') osalta lopullisten voittojen
  saavuttaminen voi olla vaikeampaa, kuten Smith alkusanoissaan
  ounastelikin.
  \end{itemize}
\end{frame}

\subsection{Äänisynteesi}
\begin{frame}{Erilaisia digitaalisen äänisynteesin menetelmiä}
\begin{itemize}
\item
  Synti2:n synteesimenetelmä perustuu (tavallaan) taajuusmodulaatioon,
  jonka sovellukset tietoliikenteessä olivat arkipäivää jo siinä
  vaiheessa, kun niiden käyttöä musiikilliseen äänisynteesiin vasta
  ehdotettiin 1970-luvulla \cite{chowning1973fm}.

\item[]
  (Demotaan käytännössä, jos tekniikka antaa myöten)

\item
  Muita viime vuosituhannen loppuun mennessä keksittyjä
  äänisynteesimenetelmiä on lueteltu ja luokiteltu esimerkiksi Tolosen
  ym. raportissa \cite{TolonenEtal1998evaluation}. Tulevaisuudessa,
  tietokoneiden suorituskyvyn kasvaessa, fysikaaliseen mallinnukseen
  perustuvien menetelmien roolin voisi olettaa yhä kasvavan.

\end{itemize}

\end{frame}

\section{DSP}
\begin{frame}{Historiaa: Digitaalinen signaalinkäsittely}
  \begin{itemize}
    \item
  Digitaalinen äänisynteesi on yksi sovellus ja osa-alue laajemmasta
  tutkimuskentästä nimeltä digitaalinen signaalinkäsittely
  (engl. digital signal processing, DSP), jossa yleisesti ottaen
  vieläkin laajemman alan, signaalinkäsittelyn, ongelmia pyritään
  ratkaisemaan digitaalitekniikan, ml. tietokoneohjelmien,
  avulla.
\item
  IEEE:n historiikista \cite{nebeker1998signalprocessing} löytyvän
  haastattelusitaatin mukaan itse asiassa koko signaalinkäsittelyn ala
  sai nimensä vasta kun tietokoneet pistivät siihen vauhtia; siihen
  asti tutkimus oli tapahtunut erillisinä saarekkeina sovellusalojen
  sisällä.

\item
  Yhtenä DSP-alan kivijalkana on mm. Shannonin informaation siirron
  matemaattisia perusteita koskeva julkaisu
  \cite{shannon48amathematical} vuodelta 1948.

\item
  Muita alan sovelluksia kuin musiikki-instrumentit? Puhelinyhteydet,
  televisio- ja radiolähetykset, tallennusvälineet, autojen
  moottoreita ohjaavat mittarit, lääketieteellinen kuvantaminen,
  seismologia, eksoplaneettojen etsiminen \ldots lisätietoa esim. em.
  historiikissa.
  \end{itemize}
  
\end{frame}

\subsection{Suotimet}
\begin{frame}{Suotimet ja suodinsuunnittelu}

  \begin{itemize}
    \item
  Esimerkki digitaalisen signaalinkäsittelyn kenttään liittyvästä
  erityiskohteesta on suodinsuunnittelu (filter design).
    \item
  Synti2:n suodin (on tietysti vaan copy-pastattu nettifoorumilta,
  mutta jos haluaisi tehdä aiheesta tiedettä, niin tulisi tietää, että
  se\ldots) perustuu Dattorron artikkeliin \cite{dattorro1997effect},
  joka edistää 1980-luvulla julkaistun kirjan \cite{Chamberlin1980}
  ideaa.
  \end{itemize}

\end{frame}

\subsection{FFT}

\begin{frame}{Taajuusspektri ja Fourier-muunnos}
  \begin{itemize}
\item
  Tärkeä ominaisuus esimerkiksi musiikissa on että suotimien ja muiden
  synteesimenetelmän osien tuottaman äänen ns. harmoninen sisältö eli
  taajuusalueen spektri saadaan sopivaksi.
\item
  Taajuuksia päästään
  tutkimaan (ja tarvittaessa manipuloimaan) muuntamalla ajan suhteen
  näytteistetty digitaalinen signaali esimerkiksi diskreetillä
  Fourier-muunnoksella (engl. discrete Fourier transform,
  DFT).
\item
  Matemaattisesti sanoen: signaalia kuvaava vektori siirretään
  toiseen avaruuteen, jonka kanta on helpommin tulkittavissa
  tarkoituksenmukaisella tavalla, eli DFT:n tapauksessa
  taajuuskomponentteina.
\item[]
  (Demotaan käytännössä, jos tekniikka antaa myöten)
  \end{itemize}
  
\end{frame}

\begin{frame}{Nopea Fourier-muunnos}
\begin{itemize}
\item
  Cooleyn ja Tukeyn julkaisu \cite{CooleyTukey1965fft} teki tunnetuksi
  nopean menetelmän diskreettien signaalien taajuuskomponenttien
  analysoimiseksi, ns. nopean Fourier-muunnoksen (fast Fourier transform)
  FFT.
\item
  Algoritmin historian seuraaminen johtaa kuitenkin jo Carl
  Friedrich Gaussin tuntemiin menettelyihin
  \footnote{\url{http://music.columbia.edu/cmc/musicandcomputers/popups/chapter3/xbit_3_2.php}}.
\item
  Nopean fourier-muunnoksen sovelluksena saadaan (matematiikkaan
  perustuen, luonnollisesti) mm. nopeat konvoluutiot. Näillä voi tehdä
  ihmeitä myös audio- ja musiikkimaailmassa..)
\item[] (Demotaan käytännössä, jos tekniikka antaa myöten)\\ Eli
  haetaan Internetin ihmeellisestä maailmasta jostain mukavasta
  mestasta mitattu impulssivaste ja ''mennään paikan päälle'' ääniaistimuksen
  mielessä\ldots Lisää aiheesta kurssijatkumossa Numeeriset
  menetelmät, numeerinen lineaarialgebra, signaalinkäsittely.
\end{itemize}
\end{frame}


\begin{frame}{Pointti: Keskittykää kursseihin alusta asti!}
\begin{itemize}
\item
  Hyödyllisen algoritmin vaativuusluokan pudottaminen luokasta
  $O(N^2)$ luokkaan $O(N \mathop{log} N)$ takaa tietoteknikolle
  ikuisen paikan historian kirjoissa! Mitä edellinen lause tarkoittaa,
  selviää kursseilla Algoritmit 1 \& Algoritmit 2.
\item
  Pitäkää mielessä sovellukset ja tavoitteet jo alkupään opinnoissa,
  vaikka teoriapainotteisilla kursseilla ei niitä kaikkia ehditä
  esittelemään.
\item Alkupään perusteoria ei vielä edes mahdollista kaikkia
  sovelluksia -- mutta edistyneemmät asiat voi ymmärtää vain aiempien
  pohjalta!
\end{itemize}
\end{frame}

\begin{frame}{Yleinen teoria mahdollistaa uudet tulokset}
\begin{itemize}
\item
  Erilaiset wavelet-muunnokset \cite{daubechies1990wavelet} ovat
  Fourier-muunnoksen ohella toinen, tietyissä tilanteissa
  tarkoituksenmukainen tapa siirtää diskreetti signaalivektori
  avaruuteen, jonka kanta on alkuperäistä
  ''helpompi''. Reunahuomautus: yhtymäkohtana aiempaan aiheeseen eli
  suotimiin, Wavelet-muunnos voidaan tulkita tietyllä tapaa
  järjestelmällisesti suunniteltujen yli- ja alipäästösuotimien
  sovelluksena.
\item
  Kaikkien näiden asioiden äärelle päästään matematiikan
  ''peruskursseilla'' eli noin kolmannen yliopistovuoden syventävillä
  kursseilla, kun niitä on ensin ehditty pohjustaa perustiedoilla
  esimerkiksi vektoriavaruuksista ja niiden kannoista sekä aivan
  kaiken taustalla olevilla funktioiden ja raja-arvojen käsitteillä.
\end{itemize}
\end{frame}

\begin{frame}{Pointti: Matematiikasta on hyväksyttävää innostua!}

\begin{itemize}
\item
  Kanditutkintoon mahtuu helposti ainakin matematiikan perusopinnot
  (25op eli melkein 700 tuntia aikaa opetella matematiikan eri
  osa-alueita.)
\item
  Käyttäkää hyvät ihmiset hyödyksi tuo aika, joka on sitä varten nyt
  annettu\ldots työelämässä voi sitten päivittää ja ylläpitää tietoja,
  mutta perustaitoja ei enää välttämättä ehdi siinä vaiheessa haalia
  alusta alkaen.
\item
  Mikäli esimerkiksi äänen- ja kuvankäsittelyn teknologiat alkavat
  kiinnostaa, ei ole paha ottaa jopa perus- ja aineopintojen
  pakettia (60 op on jo yli 1600 tuntia ihanaa matematiikka-aikaa!)
\end{itemize}
\end{frame}




\section{Nykypäivää}

\begin{frame}{Tutkimuksen laajuus tänään?}
  \begin{itemize}
  \item
    Hakusanalla ''sound synthesis'' löytyy Thomson Reuters Web of
    Knowledgesta 2010-luvulla julkaistuja lehtiartikkeleita 125
    kpl. Esimerkiksi ehdotus kvanttihiukkasen käyttäytymiseen
    perustuvasta synteesimenetelmästä
    \cite{CadizRamos2014quantum}. Ideat saavat olla villejä ja kaukaa
    haettuja!
  \item
    Hakusanalla ''filter design'' 2010-luvulla: 1466
    kpl. Musiikkisoitinten erityistarpeisiin keskittyy
    esim. \cite{Wishnick14timevaryingfilters}.
  \item
    ''digital signal processing'' 2010-luvulla: 1392 kpl.
  \item
    (pelkkä''signal processing'' 2010-luvulla: 22019 kpl.)
  \item
    ''Signal processing AND music'' 2010-luvulla: 303 kpl.
    
    Tällä haulla löytyy esimerkiksi Daviesin ja kumppaneiden juttu
    automaattisesta Mash-up -generoinnista
    \cite{daviesEtal2014automashupper}. Suomalaisen
    julkaisuluokituksen
    kakkostasolla\footnote{{\tiny \url{http://www.tsv.fi/julkaisufoorumi/haku.php?nimeke=ieee+transactions+on+audio&konferenssilyh=&issn=&tyyppi=kaikki&kieli=&maa=&wos=&scopus=&nappi=Hae}}}
    voi siis tehdä tutkimusta periaatteessa aika hauskoista asioista.

  \end{itemize}
\end{frame}


\section{Lähteitä}

\begin{frame}{Pointti: Opetelkaa hakemaan tietoa laajasti (mutta hallitusti)}
  \begin{itemize}
  \item
    Papereita julkaistaan nykyään tosi paljon, ja julkaisut löytyvät
    tietokannoista helposti. 
  \item Yliopisto maksaa opiskelijoilleen pääsyn aika kalliisiinkin
    tietolähteisiin. Hyödyntäkää etuoikeuttanne, heti tänään!
  \item Perusasiat on kaluttu kauan sitten, ja
    koko ajan ollaan uuden äärellä. On tärkeätä kyetä kartoittamaan ja
    valitsemaan relevantit julkaisut.
  \item Itseä kiinnostava tutkimusaihe voi löytyä helpommin, kun on jo
    selaillut, mitä muut ovat aiemmin kirjoittaneet aihepiirin
    ympäriltä. Huom: historia alkaa vuosikymmenten takaa!
  \item Tietoa tulee niin paljon ja monesta tuutista, että esimerkiksi
    ensimmäiset Google-osumat saattavat olla yhtä tyhjän
    kanssa. Lähtökohta voi toki löytyä\ldots Tätä luentoa varten hain
    ensimmäiseksi ''History of digital signal processing'', mutta
    vasta pykälää syvemmältä alkoi löytyä hyödyllisempiä lähteitä.
  \end{itemize}
\end{frame}

\beamerdefaultoverlayspecification{}
\begin{frame}[allowframebreaks]{Lähteitä}

  \tiny
  \bibliographystyle{plain}
  \bibliography{dspstuff}

\end{frame}

\beamerdefaultoverlayspecification{<+->}

\section{Yhteenveto}
\begin{frame}{Pointit vielä kerran}
\begin{itemize}
\item Harrastakaa vapaasti! (Oma erityisosaaminen, eli tietotekniikka,
  on parhaimmillaan mukana, mutta kaikki sovellukset sementin
  valamisesta ja komeettojen paikantamisesta optiohinnoitteluun ovat
  mahdollista pelikenttää)
\item Keskittykää kursseihin alusta asti! (Kaikki liittyy kaikkeen;
  luottakaa siihen, että kaikista opituista asioista on hyötyä, kun
  ''hedelmät kerätään'' lopulta)
\item Matematiikasta on hyväksyttävää innostua! (Luonnontieteiden ja
  tekniikan kivijalan osaamalla avaat itsellesi ovia ja erotut
  eduksesi tilanteissa, joissa ''rivikoodari'' ei riitä.)
\item Opetelkaa hakemaan tietoa! (sitä on tarjolla ähkyksi asti;
  kursseilla ei voida käydä läpi moniakaan sovelluksia, joten juuri
  omaasi ei välttämättä ehditä mainita)
  
\end{itemize}
\end{frame}

\begin{frame}{Esitys on loppu}
  Kysykää aina, ja viimeistään nyt!
\end{frame}

\end{document}
